% Options for packages loaded elsewhere
\PassOptionsToPackage{unicode}{hyperref}
\PassOptionsToPackage{hyphens}{url}
\documentclass[
]{article}
\usepackage{xcolor}
\usepackage[margin=1in]{geometry}
\usepackage{amsmath,amssymb}
\setcounter{secnumdepth}{-\maxdimen} % remove section numbering
\usepackage{iftex}
\ifPDFTeX
  \usepackage[T1]{fontenc}
  \usepackage[utf8]{inputenc}
  \usepackage{textcomp} % provide euro and other symbols
\else % if luatex or xetex
  \usepackage{unicode-math} % this also loads fontspec
  \defaultfontfeatures{Scale=MatchLowercase}
  \defaultfontfeatures[\rmfamily]{Ligatures=TeX,Scale=1}
\fi
\usepackage{lmodern}
\ifPDFTeX\else
  % xetex/luatex font selection
\fi
% Use upquote if available, for straight quotes in verbatim environments
\IfFileExists{upquote.sty}{\usepackage{upquote}}{}
\IfFileExists{microtype.sty}{% use microtype if available
  \usepackage[]{microtype}
  \UseMicrotypeSet[protrusion]{basicmath} % disable protrusion for tt fonts
}{}
\makeatletter
\@ifundefined{KOMAClassName}{% if non-KOMA class
  \IfFileExists{parskip.sty}{%
    \usepackage{parskip}
  }{% else
    \setlength{\parindent}{0pt}
    \setlength{\parskip}{6pt plus 2pt minus 1pt}}
}{% if KOMA class
  \KOMAoptions{parskip=half}}
\makeatother
\usepackage{color}
\usepackage{fancyvrb}
\newcommand{\VerbBar}{|}
\newcommand{\VERB}{\Verb[commandchars=\\\{\}]}
\DefineVerbatimEnvironment{Highlighting}{Verbatim}{commandchars=\\\{\}}
% Add ',fontsize=\small' for more characters per line
\usepackage{framed}
\definecolor{shadecolor}{RGB}{248,248,248}
\newenvironment{Shaded}{\begin{snugshade}}{\end{snugshade}}
\newcommand{\AlertTok}[1]{\textcolor[rgb]{0.94,0.16,0.16}{#1}}
\newcommand{\AnnotationTok}[1]{\textcolor[rgb]{0.56,0.35,0.01}{\textbf{\textit{#1}}}}
\newcommand{\AttributeTok}[1]{\textcolor[rgb]{0.13,0.29,0.53}{#1}}
\newcommand{\BaseNTok}[1]{\textcolor[rgb]{0.00,0.00,0.81}{#1}}
\newcommand{\BuiltInTok}[1]{#1}
\newcommand{\CharTok}[1]{\textcolor[rgb]{0.31,0.60,0.02}{#1}}
\newcommand{\CommentTok}[1]{\textcolor[rgb]{0.56,0.35,0.01}{\textit{#1}}}
\newcommand{\CommentVarTok}[1]{\textcolor[rgb]{0.56,0.35,0.01}{\textbf{\textit{#1}}}}
\newcommand{\ConstantTok}[1]{\textcolor[rgb]{0.56,0.35,0.01}{#1}}
\newcommand{\ControlFlowTok}[1]{\textcolor[rgb]{0.13,0.29,0.53}{\textbf{#1}}}
\newcommand{\DataTypeTok}[1]{\textcolor[rgb]{0.13,0.29,0.53}{#1}}
\newcommand{\DecValTok}[1]{\textcolor[rgb]{0.00,0.00,0.81}{#1}}
\newcommand{\DocumentationTok}[1]{\textcolor[rgb]{0.56,0.35,0.01}{\textbf{\textit{#1}}}}
\newcommand{\ErrorTok}[1]{\textcolor[rgb]{0.64,0.00,0.00}{\textbf{#1}}}
\newcommand{\ExtensionTok}[1]{#1}
\newcommand{\FloatTok}[1]{\textcolor[rgb]{0.00,0.00,0.81}{#1}}
\newcommand{\FunctionTok}[1]{\textcolor[rgb]{0.13,0.29,0.53}{\textbf{#1}}}
\newcommand{\ImportTok}[1]{#1}
\newcommand{\InformationTok}[1]{\textcolor[rgb]{0.56,0.35,0.01}{\textbf{\textit{#1}}}}
\newcommand{\KeywordTok}[1]{\textcolor[rgb]{0.13,0.29,0.53}{\textbf{#1}}}
\newcommand{\NormalTok}[1]{#1}
\newcommand{\OperatorTok}[1]{\textcolor[rgb]{0.81,0.36,0.00}{\textbf{#1}}}
\newcommand{\OtherTok}[1]{\textcolor[rgb]{0.56,0.35,0.01}{#1}}
\newcommand{\PreprocessorTok}[1]{\textcolor[rgb]{0.56,0.35,0.01}{\textit{#1}}}
\newcommand{\RegionMarkerTok}[1]{#1}
\newcommand{\SpecialCharTok}[1]{\textcolor[rgb]{0.81,0.36,0.00}{\textbf{#1}}}
\newcommand{\SpecialStringTok}[1]{\textcolor[rgb]{0.31,0.60,0.02}{#1}}
\newcommand{\StringTok}[1]{\textcolor[rgb]{0.31,0.60,0.02}{#1}}
\newcommand{\VariableTok}[1]{\textcolor[rgb]{0.00,0.00,0.00}{#1}}
\newcommand{\VerbatimStringTok}[1]{\textcolor[rgb]{0.31,0.60,0.02}{#1}}
\newcommand{\WarningTok}[1]{\textcolor[rgb]{0.56,0.35,0.01}{\textbf{\textit{#1}}}}
\usepackage{longtable,booktabs,array}
\usepackage{calc} % for calculating minipage widths
% Correct order of tables after \paragraph or \subparagraph
\usepackage{etoolbox}
\makeatletter
\patchcmd\longtable{\par}{\if@noskipsec\mbox{}\fi\par}{}{}
\makeatother
% Allow footnotes in longtable head/foot
\IfFileExists{footnotehyper.sty}{\usepackage{footnotehyper}}{\usepackage{footnote}}
\makesavenoteenv{longtable}
\usepackage{graphicx}
\makeatletter
\newsavebox\pandoc@box
\newcommand*\pandocbounded[1]{% scales image to fit in text height/width
  \sbox\pandoc@box{#1}%
  \Gscale@div\@tempa{\textheight}{\dimexpr\ht\pandoc@box+\dp\pandoc@box\relax}%
  \Gscale@div\@tempb{\linewidth}{\wd\pandoc@box}%
  \ifdim\@tempb\p@<\@tempa\p@\let\@tempa\@tempb\fi% select the smaller of both
  \ifdim\@tempa\p@<\p@\scalebox{\@tempa}{\usebox\pandoc@box}%
  \else\usebox{\pandoc@box}%
  \fi%
}
% Set default figure placement to htbp
\def\fps@figure{htbp}
\makeatother
\setlength{\emergencystretch}{3em} % prevent overfull lines
\providecommand{\tightlist}{%
  \setlength{\itemsep}{0pt}\setlength{\parskip}{0pt}}
\usepackage{booktabs}
\usepackage{longtable}
\usepackage{array}
\usepackage{multirow}
\usepackage{wrapfig}
\usepackage{float}
\usepackage{colortbl}
\usepackage{pdflscape}
\usepackage{tabu}
\usepackage{threeparttable}
\usepackage{threeparttablex}
\usepackage[normalem]{ulem}
\usepackage{makecell}
\usepackage{xcolor}
\usepackage{bookmark}
\IfFileExists{xurl.sty}{\usepackage{xurl}}{} % add URL line breaks if available
\urlstyle{same}
\hypersetup{
  pdftitle={Lista 3 Métodos Computacionais para Ciência de Dados II},
  pdfauthor={Emanuelle Oliveira, Maria Luiza Oliveira, Mariana Fleming},
  hidelinks,
  pdfcreator={LaTeX via pandoc}}

\title{Lista 3 Métodos Computacionais para Ciência de Dados II}
\author{Emanuelle Oliveira, Maria Luiza Oliveira, Mariana Fleming}
\date{}

\begin{document}
\maketitle

\section{Questão 1}\label{questuxe3o-1}

Considere \(\theta=\int_{0}^{1}e^{x}dx\). Use uma simulação de Monte
Carlo para estimar pelo método da variável antitética e também pelo
método de Monte Carlo simples. Calcule uma estimativa empírica da
redução percentual na variância ao usar a variável antitética.

\subsection{Solução}\label{soluuxe7uxe3o}

\subsubsection{Passo a passo}\label{passo-a-passo}

\begin{enumerate}
\def\labelenumi{\arabic{enumi}.}
\item
  Defina a função integrando \(g(x) = e^x\), cuja integral em \([0,1]\)
  é \(e - 1\).
\item
  Simule \(U_i \sim \text{Uniforme}(0,1)\). Esses números aleatórios
  serão usados para aproximar a esperança de \(g(U)\).
\item
  Primeiro, no método de Monte Carlo simples, calcule \(X_i = e^{U_i}\).
  A média de \(X_i\) é uma estimativa de \(\theta\) e a variância do
  estimador é \(\text{Var}_{X_i}/k\).
\item
  Já no método da variável antitética, use os pares \((U_i, 1-U_i)\)
  (negativamente correlacionados) e calcule a média
  \(Y_i = \frac{e^{U_i} + e^{1-U_i}}{2}\). A média de \(Y_i\) é uma
  estimativa alternativa com menor variância.
\item
  Calcule o erro padrão para comparar os dois métodos e demais métricas
  que julgar relevante (aqui usamos a redução percentual da variância
  através da fórmula
  \(\text{Redução} = \frac{\text{Var}_{MC} - \text{Var}_{ANT}}{\text{Var}_{MC}} \times 100\)).
\end{enumerate}

\subsubsection{Implementação do
código}\label{implementauxe7uxe3o-do-cuxf3digo}

\begin{Shaded}
\begin{Highlighting}[]
\CommentTok{\# parâmetros}
\FunctionTok{set.seed}\NormalTok{(}\DecValTok{123456789}\NormalTok{)}
\NormalTok{M }\OtherTok{\textless{}{-}} \DecValTok{200}
\NormalTok{k }\OtherTok{\textless{}{-}} \DecValTok{100000}
\NormalTok{g }\OtherTok{\textless{}{-}} \ControlFlowTok{function}\NormalTok{(x) }\FunctionTok{exp}\NormalTok{(x)}
\NormalTok{valor\_exato }\OtherTok{\textless{}{-}} \FunctionTok{exp}\NormalTok{(}\DecValTok{1}\NormalTok{) }\SpecialCharTok{{-}} \DecValTok{1}

\NormalTok{estimativas\_mc }\OtherTok{\textless{}{-}} \FunctionTok{numeric}\NormalTok{(M)}
\NormalTok{estimativas\_ant }\OtherTok{\textless{}{-}} \FunctionTok{numeric}\NormalTok{(M)}


\ControlFlowTok{for}\NormalTok{ (m }\ControlFlowTok{in} \DecValTok{1}\SpecialCharTok{:}\NormalTok{M) \{}
  \CommentTok{\# Monte Carlo simples}
\NormalTok{  U\_mc }\OtherTok{\textless{}{-}} \FunctionTok{runif}\NormalTok{(k)}
\NormalTok{  X\_mc }\OtherTok{\textless{}{-}} \FunctionTok{g}\NormalTok{(U\_mc)}
\NormalTok{  tc\_mc }\OtherTok{\textless{}{-}} \FunctionTok{mean}\NormalTok{(X\_mc) }\CommentTok{\# tc = theta chapéu}
\NormalTok{  var\_tc\_mc }\OtherTok{\textless{}{-}} \FunctionTok{var}\NormalTok{(X\_mc) }\SpecialCharTok{/}\NormalTok{ k}
\NormalTok{  ep\_tc\_mc }\OtherTok{\textless{}{-}} \FunctionTok{sqrt}\NormalTok{(var\_tc\_mc)}
\NormalTok{  estimativas\_mc[m] }\OtherTok{\textless{}{-}} \FunctionTok{mean}\NormalTok{(X\_mc)}

  \CommentTok{\# variável antitética}
\NormalTok{  U\_ant }\OtherTok{\textless{}{-}} \DecValTok{1} \SpecialCharTok{{-}}\NormalTok{ U\_mc}
\NormalTok{  Y\_ant }\OtherTok{\textless{}{-}}\NormalTok{ (}\FunctionTok{g}\NormalTok{(U\_mc) }\SpecialCharTok{+} \FunctionTok{g}\NormalTok{(U\_ant)) }\SpecialCharTok{/} \DecValTok{2}
\NormalTok{  tc\_ant }\OtherTok{\textless{}{-}} \FunctionTok{mean}\NormalTok{(Y\_ant)}
\NormalTok{  var\_tc\_ant }\OtherTok{\textless{}{-}} \FunctionTok{var}\NormalTok{(Y\_ant) }\SpecialCharTok{/}\NormalTok{ k}
\NormalTok{  ep\_tc\_ant }\OtherTok{\textless{}{-}} \FunctionTok{sqrt}\NormalTok{(var\_tc\_ant)}
\NormalTok{  estimativas\_ant[m] }\OtherTok{\textless{}{-}} \FunctionTok{mean}\NormalTok{(Y\_ant)}
\NormalTok{\}}

\NormalTok{reducao }\OtherTok{\textless{}{-}}\NormalTok{ ((var\_tc\_mc }\SpecialCharTok{{-}}\NormalTok{ var\_tc\_ant) }\SpecialCharTok{/}\NormalTok{ var\_tc\_mc) }\SpecialCharTok{*} \DecValTok{100} \CommentTok{\# em percentual}
\NormalTok{valor\_exato }\OtherTok{\textless{}{-}} \FunctionTok{exp}\NormalTok{(}\DecValTok{1}\NormalTok{) }\SpecialCharTok{{-}} \DecValTok{1}

\NormalTok{resultado }\OtherTok{\textless{}{-}} \FunctionTok{data.frame}\NormalTok{(}
\NormalTok{  Método }\OtherTok{=} \FunctionTok{c}\NormalTok{(}\StringTok{"Monte Carlo"}\NormalTok{, }\StringTok{"Antitética"}\NormalTok{),}
  \AttributeTok{Estimativa =} \FunctionTok{c}\NormalTok{(tc\_mc, tc\_ant),}
\NormalTok{  Variância }\OtherTok{=} \FunctionTok{c}\NormalTok{(var\_tc\_mc, var\_tc\_ant),}
\NormalTok{  ErroPadrão }\OtherTok{=} \FunctionTok{c}\NormalTok{(ep\_tc\_mc, ep\_tc\_ant)}
\NormalTok{)}

\FunctionTok{kable}\NormalTok{(resultado)}
\end{Highlighting}
\end{Shaded}

\begin{longtable}[]{@{}lrrr@{}}
\toprule\noalign{}
Método & Estimativa & Variância & ErroPadrão \\
\midrule\noalign{}
\endhead
\bottomrule\noalign{}
\endlastfoot
Monte Carlo & 1.717299 & 2.4e-06 & 0.0015570 \\
Antitética & 1.718438 & 0.0e+00 & 0.0001984 \\
\end{longtable}

\begin{Shaded}
\begin{Highlighting}[]
\FunctionTok{cat}\NormalTok{(}\StringTok{"}\SpecialCharTok{\textbackslash{}n}\StringTok{Valor exato:"}\NormalTok{, }\FunctionTok{round}\NormalTok{(valor\_exato, }\DecValTok{7}\NormalTok{),}
    \StringTok{"}\SpecialCharTok{\textbackslash{}n}\StringTok{Redução percentual na variância:"}\NormalTok{, }\FunctionTok{round}\NormalTok{(reducao, }\DecValTok{4}\NormalTok{), }\StringTok{"\%}\SpecialCharTok{\textbackslash{}n}\StringTok{"}\NormalTok{)}
\end{Highlighting}
\end{Shaded}

\begin{verbatim}
## 
## Valor exato: 1.718282 
## Redução percentual na variância: 98.3765 %
\end{verbatim}

\begin{Shaded}
\begin{Highlighting}[]
\FunctionTok{boxplot}\NormalTok{(estimativas\_mc, estimativas\_ant,}
        \AttributeTok{names =} \FunctionTok{c}\NormalTok{(}\StringTok{"MC Simples"}\NormalTok{, }\StringTok{"Antitética"}\NormalTok{),}
        \AttributeTok{main =} \StringTok{"Variabilidade dos Estimadores (M = 200)"}\NormalTok{,}
        \AttributeTok{ylab =} \StringTok{"Estimativas de θ"}\NormalTok{,}
        \AttributeTok{col =} \FunctionTok{c}\NormalTok{(}\StringTok{"lightblue"}\NormalTok{, }\StringTok{"lightgreen"}\NormalTok{))}
\FunctionTok{abline}\NormalTok{(}\AttributeTok{h =}\NormalTok{ valor\_exato, }\AttributeTok{col =} \StringTok{"red"}\NormalTok{, }\AttributeTok{lty =} \DecValTok{2}\NormalTok{, }\AttributeTok{lwd =} \DecValTok{2}\NormalTok{)}
\end{Highlighting}
\end{Shaded}

\pandocbounded{\includegraphics[keepaspectratio]{lista3_MCCDII_files/figure-latex/ex1-1.pdf}}

Portanto, percebemos que a técnica da variável antitética melhora a
eficiência da estimativa, uma vez que reduz a variância.

\section{Questão 2}\label{questuxe3o-2}

Suponha que Y seja uma variável aleatória normal com média 1 e variância
1, e que, condicionalmente a \(Y=y\), a variável X seja normal com média
y e variância 4. Deseja-se usar simulação para estimar eficientemente
\(\theta=P\{X>1\}\).

\subsection{\texorpdfstring{\textbf{a) Explique como obter o estimador
de simulação de Monte Carlo
simples.}}{a) Explique como obter o estimador de simulação de Monte Carlo simples.}}\label{a-explique-como-obter-o-estimador-de-simulauxe7uxe3o-de-monte-carlo-simples.}

\subsubsection{Solução}\label{soluuxe7uxe3o-1}

Pela lei da probabilidade total, a marginal de \(X\) é \(N(1,5)\), e
portanto o valor exato é \(\theta = P(X > 1) = 0.5\).O método de Monte
Carlo simples simula diretamente pares \((Y, X)\) e calcula a proporção
de vezes em que \(X > 1\).

O estimador (para \(\theta = \mathbb{E}[I(X>1)]\) é simples:
\(\hat\theta_{MC} = \frac{1}{K}\sum_{i=1}^K I(X_i > 1)\), onde
\(I(\cdot)\) é a função indicadora. Esse método é fácil, mas apresenta
alta variância porque depende da dupla simulação de \(Y\) e do erro
\(\varepsilon\).

\begin{Shaded}
\begin{Highlighting}[]
\CommentTok{\# parâmetros}
\NormalTok{K }\OtherTok{\textless{}{-}} \DecValTok{10000} \CommentTok{\# amostras por replicação}
\NormalTok{M }\OtherTok{\textless{}{-}} \DecValTok{200}  \CommentTok{\# número de replicações}
\NormalTok{mu\_Y }\OtherTok{\textless{}{-}} \DecValTok{1}
\NormalTok{sigma2\_Y }\OtherTok{\textless{}{-}} \DecValTok{1}
\NormalTok{sigma2\_XdadoY }\OtherTok{\textless{}{-}} \DecValTok{4}
\NormalTok{theta\_exato }\OtherTok{\textless{}{-}} \FloatTok{0.5}

\NormalTok{estimativas\_mc }\OtherTok{\textless{}{-}} \FunctionTok{numeric}\NormalTok{(M)}

\FunctionTok{set.seed}\NormalTok{(}\DecValTok{123456789}\NormalTok{)}

\ControlFlowTok{for}\NormalTok{ (m }\ControlFlowTok{in} \DecValTok{1}\SpecialCharTok{:}\NormalTok{M) \{}
\NormalTok{  Y }\OtherTok{\textless{}{-}} \FunctionTok{rnorm}\NormalTok{(K, }\AttributeTok{mean =}\NormalTok{ mu\_Y, }\AttributeTok{sd =} \FunctionTok{sqrt}\NormalTok{(sigma2\_Y))}
\NormalTok{  epsilon }\OtherTok{\textless{}{-}} \FunctionTok{rnorm}\NormalTok{(K, }\AttributeTok{mean =} \DecValTok{0}\NormalTok{, }\AttributeTok{sd =} \FunctionTok{sqrt}\NormalTok{(sigma2\_XdadoY))}
  
  \CommentTok{\# X condicional a Y}
\NormalTok{  X }\OtherTok{\textless{}{-}}\NormalTok{ Y }\SpecialCharTok{+}\NormalTok{ epsilon}
  
\NormalTok{  estimativas\_mc[m] }\OtherTok{\textless{}{-}} \FunctionTok{mean}\NormalTok{(X }\SpecialCharTok{\textgreater{}} \DecValTok{1}\NormalTok{)}
\NormalTok{\}}
\end{Highlighting}
\end{Shaded}

\subsection{\texorpdfstring{\textbf{b) Mostre como a esperança
condicional pode ser usada para obter um estimador
aprimorado.}}{b) Mostre como a esperança condicional pode ser usada para obter um estimador aprimorado.}}\label{b-mostre-como-a-esperanuxe7a-condicional-pode-ser-usada-para-obter-um-estimador-aprimorado.}

\subsubsection{Solução}\label{soluuxe7uxe3o-2}

Podemos reduzir a variância substituindo a indicadora \(I(X>1)\) por sua
esperança condicional dado \(Y\):

\[
g(Y) = E[I(X > 1) \mid Y] = P(X > 1 \mid Y) = 1 - \Phi!\left(\frac{1 - Y}{2}\right) = \Phi!\left(\frac{Y - 1}{2}\right)
\] Assim, o novo estimador é
\(\hat\theta_{EC} = \frac{1}{K} \sum_{i=1}^K g(Y_i)\). Faz sentido
utilizar esse novo estimador uma vez que esse método elimina a
variabilidade extra introduzida pela simulação de \(X\), e como
consequência, a variância de \(\hat\theta_{EC}\) é sempre menor ou igual
à de \(\hat\theta_{MC}\) pelo teorema de Rao-Blackwell.

\begin{Shaded}
\begin{Highlighting}[]
\NormalTok{estimativas\_ec }\OtherTok{\textless{}{-}} \FunctionTok{numeric}\NormalTok{(M)}

\ControlFlowTok{for}\NormalTok{ (m }\ControlFlowTok{in} \DecValTok{1}\SpecialCharTok{:}\NormalTok{M) \{}
\NormalTok{  Y }\OtherTok{\textless{}{-}} \FunctionTok{rnorm}\NormalTok{(K, }\AttributeTok{mean =}\NormalTok{ mu\_Y, }\AttributeTok{sd =} \FunctionTok{sqrt}\NormalTok{(sigma2\_Y))}
  
  \CommentTok{\#condicional g(Y) = P(X\textgreater{}1|Y)}
\NormalTok{  g\_Y }\OtherTok{\textless{}{-}} \FunctionTok{pnorm}\NormalTok{((Y }\SpecialCharTok{{-}} \DecValTok{1}\NormalTok{) }\SpecialCharTok{/} \DecValTok{2}\NormalTok{)}
  
\NormalTok{  estimativas\_ec[m] }\OtherTok{\textless{}{-}} \FunctionTok{mean}\NormalTok{(g\_Y)}
\NormalTok{\}}
\end{Highlighting}
\end{Shaded}

\subsection{\texorpdfstring{\textbf{c) Mostre como o estimador do item
(b) pode ser ainda mais aprimorado utilizando Y como variável de
controle.}}{c) Mostre como o estimador do item (b) pode ser ainda mais aprimorado utilizando Y como variável de controle.}}\label{c-mostre-como-o-estimador-do-item-b-pode-ser-ainda-mais-aprimorado-utilizando-y-como-variuxe1vel-de-controle.}

\subsubsection{Solução}\label{soluuxe7uxe3o-3}

Utilizando uma variável de controle, o objetivo é reduzir ainda mais a
variância usando uma variável correlacionada conhecida. Nesse caso, como
\(Y\) tem média conhecida (\(E[Y]=1\)), podemos usá-la como controle:

\[
\hat\theta_{VC} = \hat\theta_{EC} - c ( \bar Y - E[Y]),
\]

onde o coeficiente ótimo é
\(c^\ast = \frac{\text{Cov}(g(Y), Y)}{\text{Var}(Y)}\).

A variável de controle remove parte da variabilidade residual ainda
presente em \(g(Y)\): se \(Y\) estiver fortemente correlacionado com
\(g(Y)\), a redução de variância pode ser expressiva.

\begin{Shaded}
\begin{Highlighting}[]
\NormalTok{estimativas\_vc }\OtherTok{\textless{}{-}} \FunctionTok{numeric}\NormalTok{(M)}

\ControlFlowTok{for}\NormalTok{ (m }\ControlFlowTok{in} \DecValTok{1}\SpecialCharTok{:}\NormalTok{M) \{}
\NormalTok{  Y }\OtherTok{\textless{}{-}} \FunctionTok{rnorm}\NormalTok{(K, }\AttributeTok{mean =}\NormalTok{ mu\_Y, }\AttributeTok{sd =} \FunctionTok{sqrt}\NormalTok{(sigma2\_Y))}
  
  \CommentTok{\# condicional e coeficiente ótimo}
\NormalTok{  g\_Y }\OtherTok{\textless{}{-}} \FunctionTok{pnorm}\NormalTok{((Y }\SpecialCharTok{{-}} \DecValTok{1}\NormalTok{) }\SpecialCharTok{/} \DecValTok{2}\NormalTok{)}
\NormalTok{  c\_chapeu }\OtherTok{\textless{}{-}} \FunctionTok{cov}\NormalTok{(g\_Y, Y) }\SpecialCharTok{/} \FunctionTok{var}\NormalTok{(Y)}
  
\NormalTok{  Y\_barra }\OtherTok{\textless{}{-}} \FunctionTok{mean}\NormalTok{(Y)}
\NormalTok{  estimativas\_vc[m] }\OtherTok{\textless{}{-}} \FunctionTok{mean}\NormalTok{(g\_Y) }\SpecialCharTok{{-}}\NormalTok{ c\_chapeu }\SpecialCharTok{*}\NormalTok{ (Y\_barra }\SpecialCharTok{{-}}\NormalTok{ mu\_Y)}
\NormalTok{\}}
\end{Highlighting}
\end{Shaded}

\subsection{\texorpdfstring{\textbf{d) Implemente os três métodos acima
e mostre a variabilidade dos estimadores replicando o uso de cada
algoritmo 200
vezes.}}{d) Implemente os três métodos acima e mostre a variabilidade dos estimadores replicando o uso de cada algoritmo 200 vezes.}}\label{d-implemente-os-truxeas-muxe9todos-acima-e-mostre-a-variabilidade-dos-estimadores-replicando-o-uso-de-cada-algoritmo-200-vezes.}

\subsubsection{Solução}\label{soluuxe7uxe3o-4}

Podemos comparar empiricamente, através das simulações, a variabilidade,
a média estimada (viés) e a redução percentual de variância entre os
métodos.

\begin{Shaded}
\begin{Highlighting}[]
\NormalTok{ep\_mc }\OtherTok{\textless{}{-}} \FunctionTok{sd}\NormalTok{(estimativas\_mc)}
\NormalTok{ep\_ec }\OtherTok{\textless{}{-}} \FunctionTok{sd}\NormalTok{(estimativas\_ec)}
\NormalTok{ep\_vc }\OtherTok{\textless{}{-}} \FunctionTok{sd}\NormalTok{(estimativas\_vc)}

\NormalTok{media\_mc }\OtherTok{\textless{}{-}} \FunctionTok{mean}\NormalTok{(estimativas\_mc)}
\NormalTok{media\_ec }\OtherTok{\textless{}{-}} \FunctionTok{mean}\NormalTok{(estimativas\_ec)}
\NormalTok{media\_vc }\OtherTok{\textless{}{-}} \FunctionTok{mean}\NormalTok{(estimativas\_vc)}

\CommentTok{\# redução de variância (em comparação com o método simples)}
\NormalTok{reducao\_ec }\OtherTok{\textless{}{-}}\NormalTok{ (ep\_mc}\SpecialCharTok{\^{}}\DecValTok{2} \SpecialCharTok{{-}}\NormalTok{ ep\_ec}\SpecialCharTok{\^{}}\DecValTok{2}\NormalTok{) }\SpecialCharTok{/}\NormalTok{ ep\_mc}\SpecialCharTok{\^{}}\DecValTok{2} \SpecialCharTok{*} \DecValTok{100}
\NormalTok{reducao\_vc }\OtherTok{\textless{}{-}}\NormalTok{ (ep\_mc}\SpecialCharTok{\^{}}\DecValTok{2} \SpecialCharTok{{-}}\NormalTok{ ep\_vc}\SpecialCharTok{\^{}}\DecValTok{2}\NormalTok{) }\SpecialCharTok{/}\NormalTok{ ep\_mc}\SpecialCharTok{\^{}}\DecValTok{2} \SpecialCharTok{*} \DecValTok{100}

\NormalTok{resultados }\OtherTok{\textless{}{-}} \FunctionTok{data.frame}\NormalTok{(}
  \StringTok{\textasciigrave{}}\AttributeTok{Método}\StringTok{\textasciigrave{}} \OtherTok{=} \FunctionTok{c}\NormalTok{(}\StringTok{"MC Simples"}\NormalTok{, }\StringTok{"Esperança Condicional"}\NormalTok{, }\StringTok{"EC + Variável de Controle"}\NormalTok{),}
  \StringTok{\textasciigrave{}}\AttributeTok{Média Estimada}\StringTok{\textasciigrave{}} \OtherTok{=} \FunctionTok{c}\NormalTok{(media\_mc, media\_ec, media\_vc),}
  \StringTok{\textasciigrave{}}\AttributeTok{Erro Padrão Estimado}\StringTok{\textasciigrave{}} \OtherTok{=} \FunctionTok{c}\NormalTok{(ep\_mc, ep\_ec, ep\_vc),}
  \StringTok{\textasciigrave{}}\AttributeTok{Redução de Variância}\StringTok{\textasciigrave{}} \OtherTok{=} \FunctionTok{c}\NormalTok{(}\FloatTok{0.0}\NormalTok{, reducao\_ec, reducao\_vc)}
\NormalTok{)}
\FunctionTok{kable}\NormalTok{(resultados)}
\end{Highlighting}
\end{Shaded}

\begin{longtable}[]{@{}
  >{\raggedright\arraybackslash}p{(\linewidth - 6\tabcolsep) * \real{0.3133}}
  >{\raggedleft\arraybackslash}p{(\linewidth - 6\tabcolsep) * \real{0.1807}}
  >{\raggedleft\arraybackslash}p{(\linewidth - 6\tabcolsep) * \real{0.2530}}
  >{\raggedleft\arraybackslash}p{(\linewidth - 6\tabcolsep) * \real{0.2530}}@{}}
\toprule\noalign{}
\begin{minipage}[b]{\linewidth}\raggedright
Método
\end{minipage} & \begin{minipage}[b]{\linewidth}\raggedleft
Média.Estimada
\end{minipage} & \begin{minipage}[b]{\linewidth}\raggedleft
Erro.Padrão.Estimado
\end{minipage} & \begin{minipage}[b]{\linewidth}\raggedleft
Redução.de.Variância
\end{minipage} \\
\midrule\noalign{}
\endhead
\bottomrule\noalign{}
\endlastfoot
MC Simples & 0.5000730 & 0.0049881 & 0.00000 \\
Esperança Condicional & 0.5000034 & 0.0018497 & 86.24926 \\
EC + Variável de Controle & 0.5000134 & 0.0001411 & 99.92002 \\
\end{longtable}

\begin{Shaded}
\begin{Highlighting}[]
\FunctionTok{boxplot}\NormalTok{(estimativas\_mc, estimativas\_ec, estimativas\_vc,}
        \AttributeTok{names =} \FunctionTok{c}\NormalTok{(}\StringTok{"MC Simples"}\NormalTok{, }\StringTok{"E. Condicional"}\NormalTok{, }\StringTok{"EC + VC"}\NormalTok{),}
        \AttributeTok{main =} \StringTok{"Variabilidade dos Estimadores (M=200)"}\NormalTok{,}
        \AttributeTok{ylab =} \StringTok{"Estimativas de theta"}\NormalTok{,}
        \AttributeTok{col =} \FunctionTok{c}\NormalTok{(}\StringTok{"lightblue"}\NormalTok{, }\StringTok{"lightgreen"}\NormalTok{, }\StringTok{"salmon"}\NormalTok{))}
\FunctionTok{abline}\NormalTok{(}\AttributeTok{h =}\NormalTok{ theta\_exato, }\AttributeTok{col =} \StringTok{"red"}\NormalTok{, }\AttributeTok{lty =} \DecValTok{2}\NormalTok{, }\AttributeTok{lwd =} \DecValTok{2}\NormalTok{)}
\end{Highlighting}
\end{Shaded}

\pandocbounded{\includegraphics[keepaspectratio]{lista3_MCCDII_files/figure-latex/ex2d-1.pdf}}

Logo, concluímos que o estimador de Monte Carlo simples é correto, mas
ineficiente. A esperança condicional (de acordo com o teorema de
Rao-Blackwell) melhora a eficiência sem introduzir viés e a adição da
variável de controle \(Y\) produz ganhos adicionais, embora menores,
refinando o estimador.

\section{Questão 3}\label{questuxe3o-3}

Obtenha uma estimativa de Monte Carlo para
\(\int_{1}^{\infty}\frac{x^{2}}{\sqrt{2\pi}}e^{-x^{2}/2}dx\) utilizando
o método de amostragem por importância (importance sampling).

\subsection{Solução}\label{soluuxe7uxe3o-5}

No \textit{Importance Sampling} podemos reescrever a integral alvo
(\(f(x)\)) de uma forma que possamos amostrar de uma distribuição de
probabilidade proposta (\(g(x)\)) da nossa escolha, para isso
reescrevemos:

\(\int f(x)dx = \int g(x) \frac{f(x)}{g(x)}dx = \int g(x)w(x)dx\) onde
\(w(x)\) chamamos de peso.

Utilizaremos dessa integral como valor esperado da função
\(w(x) = \frac{f(x)}{g(x)}\) quando amostrarmos \(X\) da proposta
\(g(x)\):

\(I = E_g[w(x)]\)

Logo, nosso Monte Carlo para \(I\) com \(N\) amostras será a média:

\(Î = \frac{1}{N}\sum_{i=1}^{N} w(x)\)

Temos como alvo
\(f(x) = \int_{1}^{\infty}\frac{x^{2}}{\sqrt{2\pi}}e^{-x^{2}/2}dx\), que
pode ser interpretada como \(f(x) = x^2 \cdot \phi(x)\), onde
\(\phi(x) = \frac{1}{\sqrt{2\pi}}e^{-x^{2}/2}\) é a densidade da Normal
Padrão. Como proposta escolhemos a Normal Truncada no intervalo da
integral alvo, \([1, \infty)\), \(g(x)=\frac{\phi(x)}{1-\Phi(1)}\), onde
o denominador iremos chamar de constante normalizadora \(C=1-\Phi(1)\).

Reconstruindo nossa função de peso \(w(x)\) teremos:

\(w(x) = \frac{f(x)}{g(x)} = \frac{x^2 \cdot \phi(x)}{\frac{\phi(x)}{C}} = C \cdot x^2\)

Então teremos nosso estimador de Monte Carlo:

\(Î = \frac{1}{N}\sum_{i=1}^{N}C\cdot x_i^2\)

Amostra proposta \(g(x)\):

\begin{Shaded}
\begin{Highlighting}[]
\FunctionTok{set.seed}\NormalTok{(}\DecValTok{453}
\NormalTok{         )}
\NormalTok{dnormtrunc }\OtherTok{\textless{}{-}} \ControlFlowTok{function}\NormalTok{(x, }\AttributeTok{mu=}\DecValTok{0}\NormalTok{, }\AttributeTok{sigma=}\DecValTok{1}\NormalTok{, a, b)\{}
\NormalTok{  d }\OtherTok{\textless{}{-}} \FunctionTok{dnorm}\NormalTok{(x, mu, sigma)}\SpecialCharTok{/}\NormalTok{( }\FunctionTok{pnorm}\NormalTok{(b, mu, sigma) }\SpecialCharTok{{-}} \FunctionTok{pnorm}\NormalTok{(a, mu, sigma) )}
\NormalTok{\}}

\NormalTok{rnormtrunc }\OtherTok{\textless{}{-}} \ControlFlowTok{function}\NormalTok{(n, }\AttributeTok{mu=}\DecValTok{0}\NormalTok{, }\AttributeTok{sigma=}\DecValTok{1}\NormalTok{, a, b)\{}
\NormalTok{  us }\OtherTok{\textless{}{-}} \FunctionTok{runif}\NormalTok{(n)}
\NormalTok{  p1 }\OtherTok{\textless{}{-}} \FunctionTok{pnorm}\NormalTok{(a, mu, sigma)}
\NormalTok{  p2 }\OtherTok{\textless{}{-}} \FunctionTok{pnorm}\NormalTok{(b, mu, sigma)}
\NormalTok{  p }\OtherTok{\textless{}{-}}\NormalTok{ p1 }\SpecialCharTok{+}\NormalTok{ (p2}\SpecialCharTok{{-}}\NormalTok{p1)}\SpecialCharTok{*}\NormalTok{us}
\NormalTok{  amostra }\OtherTok{\textless{}{-}} \FunctionTok{qnorm}\NormalTok{( p, }\AttributeTok{mean=}\NormalTok{mu, }\AttributeTok{sd=}\NormalTok{sigma )}
\NormalTok{\}}

\NormalTok{a}\OtherTok{=}\DecValTok{1}
\NormalTok{b}\OtherTok{=}\DecValTok{1000000} \CommentTok{\#Aproximacao de um numero muito grande como "infinito"}
\NormalTok{amostra\_proposta }\OtherTok{\textless{}{-}} \FunctionTok{rnormtrunc}\NormalTok{(}\DecValTok{100000}\NormalTok{, }\AttributeTok{a=}\NormalTok{a, }\AttributeTok{b=}\NormalTok{b)}

\FunctionTok{par}\NormalTok{(}\AttributeTok{mar =} \FunctionTok{c}\NormalTok{(}\DecValTok{4}\NormalTok{, }\DecValTok{4}\NormalTok{, }\DecValTok{1}\NormalTok{, }\DecValTok{1}\NormalTok{))}
\FunctionTok{hist}\NormalTok{(amostra\_proposta, }\AttributeTok{col =} \StringTok{"lightblue"}\NormalTok{, }\AttributeTok{xlab =} \FunctionTok{expression}\NormalTok{(x), }\AttributeTok{ylab =} \StringTok{"Densidade"}\NormalTok{, }
     \AttributeTok{prob =}\NormalTok{ T, }\AttributeTok{main =} \StringTok{""}\NormalTok{)}
\NormalTok{grid }\OtherTok{\textless{}{-}} \FunctionTok{seq}\NormalTok{(a, b, }\AttributeTok{by =} \FloatTok{0.01}\NormalTok{)}
\FunctionTok{lines}\NormalTok{(grid, }\FunctionTok{dnormtrunc}\NormalTok{(grid, }\AttributeTok{a=}\NormalTok{a, }\AttributeTok{b=}\NormalTok{b), }\AttributeTok{col =} \StringTok{"red"}\NormalTok{, }\AttributeTok{lwd =} \DecValTok{2}\NormalTok{ )}
\end{Highlighting}
\end{Shaded}

\pandocbounded{\includegraphics[keepaspectratio]{lista3_MCCDII_files/figure-latex/amostra-proposta-sir-1.pdf}}

Simulação de Monte Carlo:

\begin{Shaded}
\begin{Highlighting}[]
\NormalTok{N }\OtherTok{\textless{}{-}} \DecValTok{100000}
\NormalTok{C }\OtherTok{\textless{}{-}} \DecValTok{1} \SpecialCharTok{{-}} \FunctionTok{pnorm}\NormalTok{(}\DecValTok{1}\NormalTok{)}
\NormalTok{amostras\_x }\OtherTok{\textless{}{-}} \FunctionTok{rnormtrunc}\NormalTok{(}\AttributeTok{n =}\NormalTok{ N, }\AttributeTok{mu =} \DecValTok{0}\NormalTok{, }\AttributeTok{sigma =} \DecValTok{1}\NormalTok{, }\AttributeTok{a =} \DecValTok{1}\NormalTok{, }\AttributeTok{b =} \ConstantTok{Inf}\NormalTok{)}

\NormalTok{estimativa\_I }\OtherTok{\textless{}{-}} \FunctionTok{mean}\NormalTok{(C }\SpecialCharTok{*}\NormalTok{ amostras\_x}\SpecialCharTok{\^{}}\DecValTok{2}\NormalTok{)}
\NormalTok{valor\_exato }\OtherTok{\textless{}{-}} \FunctionTok{dnorm}\NormalTok{(}\DecValTok{1}\NormalTok{) }\SpecialCharTok{+}\NormalTok{ (}\DecValTok{1} \SpecialCharTok{{-}} \FunctionTok{pnorm}\NormalTok{(}\DecValTok{1}\NormalTok{))}

\NormalTok{tabela }\OtherTok{\textless{}{-}} \FunctionTok{data.frame}\NormalTok{(}
  \AttributeTok{Metodo =} \FunctionTok{c}\NormalTok{(}\StringTok{"Estimativa de Monte Carlo"}\NormalTok{, }\StringTok{"Valor Analitico"}\NormalTok{),}
  \AttributeTok{Valor =} \FunctionTok{c}\NormalTok{(estimativa\_I, valor\_exato)}
\NormalTok{)}

\FunctionTok{kable}\NormalTok{(tabela, }
      \AttributeTok{caption =} \StringTok{"Metodo para Integral"}\NormalTok{,}
      \AttributeTok{booktabs =}\NormalTok{ T) }\SpecialCharTok{\%\textgreater{}\%}
    \FunctionTok{kable\_styling}\NormalTok{(}\AttributeTok{bootstrap\_options =} \FunctionTok{c}\NormalTok{(}\StringTok{"simple"}\NormalTok{, }\StringTok{"hover"}\NormalTok{), }
                  \AttributeTok{full\_width =}\NormalTok{ F,}
                  \AttributeTok{position =} \StringTok{"center"}\NormalTok{)}
\end{Highlighting}
\end{Shaded}

\begin{longtable}[t]{lr}
\caption{\label{tab:monte-carlo-sir}Metodo para Integral}\\
\toprule
Metodo & Valor\\
\midrule
Estimativa de Monte Carlo & 0.4001593\\
Valor Analitico & 0.4006260\\
\bottomrule
\end{longtable}

\section{Questão 4}\label{questuxe3o-4}

Utilize o algoritmo de Newton-Raphson para estimar os parâmetros de um
modelo de regressão Poisson em que
\(y_{i}\sim Poisson(\lambda=exp\{\beta_{0}+\beta_{1}x_{i}\})\). Gere
dados do modelo de regressão e avalie se o algoritmo fornece boas
estimativas.

\subsection{Solução}\label{soluuxe7uxe3o-6}

Para estimar os parâmetros \(\beta_0\) e \(\beta_1\) de uma regressão
Poisson, precisamos encontrar os valores que maximizam a função de
log-verossimilhança \((l(\beta))\). Nesse caso procuramos o ponto onde a
primeira derivada de \(f(x)\) é igual a zero.

Como o problema é multidimensional (temos dois \(\beta\)s), nossas
ferramentas são:

\begin{itemize}
\item
  A 1º derivada \(g'(x)\): Vetor Score \(U(\beta)\) com as derivadas
  parciais de \(l(\beta)\) em relação a cada \(\beta\). Nosso objetivo é
  encontrar o \(\beta\) que faz \(U(\beta) = 0\).
\item
  A 2º derivada \(g''(x)\): Matriz Hessiana \(H(\beta)\). É uma matriz
  2x2 com todas as segundas derivadas.
\end{itemize}

Usando a fórmula de atualização de Newton-Raphson
\[\beta^{(t+1)} = \beta^{(t)} - [H(\beta^{(t)}]^{-1}U(\beta^{(t)})\].
Onde \([H^{-1}]\) é a matriz inversa hessiana.

Para implementar, definimos os componentes do nosso GLM Poisson. O
preditor linear é \(\eta = X\beta\). A média da Poisson, \(\lambda\), é
ligada ao preditor pela função de ligação log, \(\log(\lambda) = \eta\).

Invertendo a ligação, temos \(\lambda = \exp(\eta)\). Esta média
\(\lambda\) é usada tanto no Vetor Score (\(U\)) quanto na Matriz
Hessiana (\(H\)). A Hessiana é calculada como \(H = -X^T W X\), onde
\(W\) é uma matriz diagonal de pesos com \(W_{ii} = \lambda_i\).

\textbf{Algoritmo}

\begin{enumerate}
\def\labelenumi{\arabic{enumi})}
\item
  Definir um chute inicial \(\beta^{(0)}\)
\item
  Loop de Iterações:
\end{enumerate}

- calcular \(\eta^{(t)} = X\beta^{(t)}\) e
\(\lambda^{(t)} = exp(\eta^{(t)})\)

- calcular o \textbf{vetor score} \(U^{(t)} = X^T(y - \lambda^{(t)})\)

- calcular a \textbf{matriz
hessiana}:\(H^{(t)} = -\mathbf{X}^T \mathbf{W}^{(t)} \mathbf{X}\)

- calcular o passo: \(\Delta = [H^{(t)}]^{-1} U^{(t)}\)

- atualizar o chute: \(\beta^{(t+1)} = \beta^{(t)} - \Delta\)

- Se a soma das mudanças for menor que a tolerância, paramos o loop.

\begin{Shaded}
\begin{Highlighting}[]
\FunctionTok{set.seed}\NormalTok{(}\DecValTok{42}\NormalTok{)}

\NormalTok{n }\OtherTok{\textless{}{-}} \DecValTok{1000} \CommentTok{\# tamanho da amostra}
\NormalTok{beta\_real }\OtherTok{\textless{}{-}} \FunctionTok{c}\NormalTok{(}\FloatTok{1.5}\NormalTok{, }\SpecialCharTok{{-}}\FloatTok{0.8}\NormalTok{) }\CommentTok{\# parâmetros verdadeiros}

\NormalTok{x }\OtherTok{\textless{}{-}} \FunctionTok{runif}\NormalTok{(n, }\AttributeTok{min =} \SpecialCharTok{{-}}\DecValTok{3}\NormalTok{, }\AttributeTok{max =} \DecValTok{3}\NormalTok{) }
\NormalTok{X }\OtherTok{\textless{}{-}} \FunctionTok{cbind}\NormalTok{(}\FunctionTok{rep}\NormalTok{(}\DecValTok{1}\NormalTok{, n), x) }\CommentTok{\# Matriz de desenho}

\CommentTok{\# Calcular o preditor linear e a média }
\NormalTok{eta\_real }\OtherTok{\textless{}{-}}\NormalTok{ X }\SpecialCharTok{\%*\%}\NormalTok{ beta\_real   }\CommentTok{\# eta = beta\_0*1 + beta\_1*x\_i}
\NormalTok{lambda\_real }\OtherTok{\textless{}{-}} \FunctionTok{exp}\NormalTok{(eta\_real)  }\CommentTok{\# lambda = exp(eta)}

\CommentTok{\# Gerar os dados da distribuição Poisson}
\NormalTok{y }\OtherTok{\textless{}{-}} \FunctionTok{rpois}\NormalTok{(n, }\AttributeTok{lambda =}\NormalTok{ lambda\_real)}

\CommentTok{\# Chute inicial para os parâmetros}
\NormalTok{beta\_estimado }\OtherTok{\textless{}{-}} \FunctionTok{c}\NormalTok{(}\DecValTok{0}\NormalTok{, }\DecValTok{0}\NormalTok{)}

\NormalTok{tolerancia }\OtherTok{\textless{}{-}} \FloatTok{1e{-}8}      \CommentTok{\# Critério de parada}
\NormalTok{max\_iteracoes }\OtherTok{\textless{}{-}} \DecValTok{100}  \CommentTok{\# Limite de segurança}


\ControlFlowTok{for}\NormalTok{ (i }\ControlFlowTok{in} \DecValTok{1}\SpecialCharTok{:}\NormalTok{max\_iteracoes) \{}
  
  \CommentTok{\# Calcular eta e lambda com o beta ATUAL}
\NormalTok{  eta\_atual }\OtherTok{\textless{}{-}}\NormalTok{ X }\SpecialCharTok{\%*\%}\NormalTok{ beta\_estimado}
\NormalTok{  lambda\_atual }\OtherTok{\textless{}{-}} \FunctionTok{exp}\NormalTok{(eta\_atual)}
  
  \CommentTok{\# Calcular o Vetor Score (U)}
\NormalTok{  score }\OtherTok{\textless{}{-}} \FunctionTok{t}\NormalTok{(X) }\SpecialCharTok{\%*\%}\NormalTok{ (y }\SpecialCharTok{{-}}\NormalTok{ lambda\_atual)}
  
  \CommentTok{\# 4c. Calcular a Matriz Hessiana (H)}
\NormalTok{  W }\OtherTok{\textless{}{-}} \FunctionTok{diag}\NormalTok{(}\FunctionTok{as.vector}\NormalTok{(lambda\_atual)) }\CommentTok{\# matriz diagonal de pesos (lambdas)}
\NormalTok{  hessiana }\OtherTok{\textless{}{-}} \SpecialCharTok{{-}}\FunctionTok{t}\NormalTok{(X) }\SpecialCharTok{\%*\%}\NormalTok{ W }\SpecialCharTok{\%*\%}\NormalTok{ X}
  
  \CommentTok{\# Calcular o passo de Newton = H\^{}({-}1) * U}
 
\NormalTok{  passo }\OtherTok{\textless{}{-}} \FunctionTok{solve}\NormalTok{(hessiana, score)  }\CommentTok{\# usamos solve(A, b) para calcular A\^{}({-}1) * b}
  
  \CommentTok{\# Atualizar os parâmetros: beta\_novo = beta\_antigo {-} passo}
\NormalTok{  beta\_novo }\OtherTok{\textless{}{-}}\NormalTok{ beta\_estimado }\SpecialCharTok{{-}}\NormalTok{ passo}
  
  \CommentTok{\# Checar a convergência. Se a mudança for muito pequena, paramos o loop}
\NormalTok{  mudanca }\OtherTok{\textless{}{-}} \FunctionTok{sum}\NormalTok{(}\FunctionTok{abs}\NormalTok{(passo))}
  \ControlFlowTok{if}\NormalTok{ (mudanca }\SpecialCharTok{\textless{}}\NormalTok{ tolerancia) \{}
    \FunctionTok{print}\NormalTok{(}\FunctionTok{paste}\NormalTok{(}\StringTok{"Convergência alcançada na iteração:"}\NormalTok{, i))}
\NormalTok{    beta\_estimado }\OtherTok{\textless{}{-}}\NormalTok{ beta\_novo}
    \ControlFlowTok{break}
\NormalTok{  \}}
  
  \CommentTok{\# Se não convergiu, atualiza o chute para a próxima rodada}
\NormalTok{  beta\_estimado }\OtherTok{\textless{}{-}}\NormalTok{ beta\_novo}
  
\NormalTok{\}}
\end{Highlighting}
\end{Shaded}

\begin{verbatim}
## [1] "Convergência alcançada na iteração: 29"
\end{verbatim}

\begin{Shaded}
\begin{Highlighting}[]
\FunctionTok{print}\NormalTok{(}\StringTok{"Parâmetros Reais:"}\NormalTok{)}
\end{Highlighting}
\end{Shaded}

\begin{verbatim}
## [1] "Parâmetros Reais:"
\end{verbatim}

\begin{Shaded}
\begin{Highlighting}[]
\NormalTok{beta\_real}
\end{Highlighting}
\end{Shaded}

\begin{verbatim}
## [1]  1.5 -0.8
\end{verbatim}

\begin{Shaded}
\begin{Highlighting}[]
\FunctionTok{print}\NormalTok{(}\StringTok{"Parâmetros Estimados (Newton{-}Raphson):"}\NormalTok{)}
\end{Highlighting}
\end{Shaded}

\begin{verbatim}
## [1] "Parâmetros Estimados (Newton-Raphson):"
\end{verbatim}

\begin{Shaded}
\begin{Highlighting}[]
\FunctionTok{as.vector}\NormalTok{(beta\_estimado)}
\end{Highlighting}
\end{Shaded}

\begin{verbatim}
## [1]  1.5263798 -0.7940793
\end{verbatim}

\subsubsection{Gráfico para
comparação}\label{gruxe1fico-para-comparauxe7uxe3o}

\begin{Shaded}
\begin{Highlighting}[]
\FunctionTok{plot}\NormalTok{(x, y, }\AttributeTok{main =} \StringTok{"Regressão Poisson com Newton{-}Raphson"}\NormalTok{,}
     \AttributeTok{xlab =} \StringTok{"Variável X"}\NormalTok{, }\AttributeTok{ylab =} \StringTok{"Contagem Y"}\NormalTok{,}
     \AttributeTok{pch =} \DecValTok{20}\NormalTok{, }\AttributeTok{col =} \StringTok{"gray"}\NormalTok{,}
     \AttributeTok{ylim =} \FunctionTok{c}\NormalTok{(}\DecValTok{0}\NormalTok{, }\FunctionTok{max}\NormalTok{(y[x }\SpecialCharTok{\textless{}} \DecValTok{0}\NormalTok{]))) }

\CommentTok{\# curva real}
\FunctionTok{curve}\NormalTok{(}\FunctionTok{exp}\NormalTok{(beta\_real[}\DecValTok{1}\NormalTok{] }\SpecialCharTok{+}\NormalTok{ beta\_real[}\DecValTok{2}\NormalTok{] }\SpecialCharTok{*}\NormalTok{ x),}
      \AttributeTok{col =} \StringTok{"blue"}\NormalTok{, }\AttributeTok{lty =} \DecValTok{2}\NormalTok{, }\AttributeTok{lwd =} \DecValTok{3}\NormalTok{, }\AttributeTok{add =} \ConstantTok{TRUE}\NormalTok{)}

\CommentTok{\# curva estimada}
\FunctionTok{curve}\NormalTok{(}\FunctionTok{exp}\NormalTok{(beta\_estimado[}\DecValTok{1}\NormalTok{] }\SpecialCharTok{+}\NormalTok{ beta\_estimado[}\DecValTok{2}\NormalTok{] }\SpecialCharTok{*}\NormalTok{ x),}
      \AttributeTok{col =} \StringTok{"red"}\NormalTok{, }\AttributeTok{lty =} \DecValTok{1}\NormalTok{, }\AttributeTok{lwd =} \DecValTok{3}\NormalTok{, }\AttributeTok{add =} \ConstantTok{TRUE}\NormalTok{)}

\FunctionTok{legend}\NormalTok{(}\StringTok{"topright"}\NormalTok{,}
       \AttributeTok{legend =} \FunctionTok{c}\NormalTok{(}\StringTok{"Dados Gerados"}\NormalTok{, }\StringTok{"Curva Real"}\NormalTok{, }\StringTok{"Curva Estimada"}\NormalTok{),}
       \AttributeTok{col =} \FunctionTok{c}\NormalTok{(}\StringTok{"gray"}\NormalTok{, }\StringTok{"blue"}\NormalTok{, }\StringTok{"red"}\NormalTok{),}
       \AttributeTok{lty =} \FunctionTok{c}\NormalTok{(}\ConstantTok{NA}\NormalTok{, }\DecValTok{2}\NormalTok{, }\DecValTok{1}\NormalTok{),}
       \AttributeTok{pch =} \FunctionTok{c}\NormalTok{(}\DecValTok{20}\NormalTok{, }\ConstantTok{NA}\NormalTok{, }\ConstantTok{NA}\NormalTok{),}
       \AttributeTok{lwd =} \FunctionTok{c}\NormalTok{(}\ConstantTok{NA}\NormalTok{, }\DecValTok{3}\NormalTok{, }\DecValTok{3}\NormalTok{))}
\end{Highlighting}
\end{Shaded}

\pandocbounded{\includegraphics[keepaspectratio]{lista3_MCCDII_files/figure-latex/unnamed-chunk-2-1.pdf}}

\section{Questão 5}\label{questuxe3o-5}

Implemente um estudo Monte Carlo com \(M=100\) réplicas para avaliar o
algoritmo EM implementado nos slides para o modelo Poisson inflado de
zeros e avalie as estimativas nos 4 cenários apresentados nos slides.

\subsection{Solução}\label{soluuxe7uxe3o-7}

\subsubsection{Algoritmo EM dos slides:}\label{algoritmo-em-dos-slides}

\begin{Shaded}
\begin{Highlighting}[]
\FunctionTok{set.seed}\NormalTok{(}\DecValTok{156}\NormalTok{)}

\CommentTok{\# Gerando dados com excesso de zero }
\NormalTok{rPoisson\_zero }\OtherTok{\textless{}{-}} \ControlFlowTok{function}\NormalTok{(n, lambda, p)\{}
\NormalTok{  s }\OtherTok{\textless{}{-}} \FunctionTok{rbinom}\NormalTok{(n, }\DecValTok{1}\NormalTok{, p) }\CommentTok{\# valor 1 indica um zero em excesso}
\NormalTok{  x }\OtherTok{\textless{}{-}} \FunctionTok{rpois}\NormalTok{(n, lambda) }\SpecialCharTok{*}\NormalTok{ (s }\SpecialCharTok{==} \DecValTok{0}\NormalTok{) }
\NormalTok{\}}

\DocumentationTok{\#\# Algoritmo EM {-} Poisson zero inflacionado}
\NormalTok{EM\_Poisson\_zinf }\OtherTok{\textless{}{-}} \ControlFlowTok{function}\NormalTok{(x, M)\{}
  \CommentTok{\# 1. Inicialize p\^{}(0) e lambda\^{}(0)}
\NormalTok{  lambda }\OtherTok{\textless{}{-}} \DecValTok{5}
\NormalTok{  p }\OtherTok{\textless{}{-}} \FloatTok{0.05}
  
\NormalTok{  lambda\_vec }\OtherTok{\textless{}{-}}\NormalTok{ lambda}
\NormalTok{  p\_vec }\OtherTok{\textless{}{-}}\NormalTok{ p}
  
  \CommentTok{\# 2. Faça t=0;}
\NormalTok{  t }\OtherTok{\textless{}{-}} \DecValTok{0}
  
  \CommentTok{\# 3. Repita os passos abaixo até atingir a convergência:}
  
  \ControlFlowTok{while}\NormalTok{(t }\SpecialCharTok{\textless{}}\NormalTok{ M)\{}
    
    \DocumentationTok{\#\#  Calcule gamma\_\{i1\}(t) e gamma\_\{i0\}(t)}
\NormalTok{    aux1 }\OtherTok{\textless{}{-}}\NormalTok{ p }\SpecialCharTok{*}\NormalTok{ (x }\SpecialCharTok{==} \DecValTok{0}\NormalTok{)}\SpecialCharTok{*}\DecValTok{1} \CommentTok{\# vetor de tamanho n}
\NormalTok{    aux2 }\OtherTok{\textless{}{-}}\NormalTok{ (}\DecValTok{1}\SpecialCharTok{{-}}\NormalTok{p) }\SpecialCharTok{*} \FunctionTok{dpois}\NormalTok{(x, lambda) }\CommentTok{\# vetor de tamanho n}
\NormalTok{    gamma1 }\OtherTok{\textless{}{-}}\NormalTok{ aux1 }\SpecialCharTok{/}\NormalTok{ (aux1 }\SpecialCharTok{+}\NormalTok{ aux2) }\CommentTok{\# vetor de tamanho n relativo a excessos de zero}
\NormalTok{    gamma0 }\OtherTok{\textless{}{-}} \DecValTok{1} \SpecialCharTok{{-}}\NormalTok{ gamma1}
    
    \DocumentationTok{\#\#  Obtenha p\^{}(t+1) e lambda\^{}(t+1) com os estimadores obtidos no passo de maximização; }
\NormalTok{    n }\OtherTok{\textless{}{-}} \FunctionTok{length}\NormalTok{(x)}
\NormalTok{    p }\OtherTok{\textless{}{-}} \FunctionTok{sum}\NormalTok{(gamma1)}\SpecialCharTok{/}\NormalTok{n}
\NormalTok{    lambda }\OtherTok{\textless{}{-}} \FunctionTok{sum}\NormalTok{(x}\SpecialCharTok{*}\NormalTok{gamma0)}\SpecialCharTok{/}\FunctionTok{sum}\NormalTok{(gamma0)}
    
\NormalTok{    p\_vec }\OtherTok{\textless{}{-}} \FunctionTok{c}\NormalTok{(p\_vec, p)}
\NormalTok{    lambda\_vec }\OtherTok{\textless{}{-}} \FunctionTok{c}\NormalTok{(lambda\_vec, lambda)}
    
    \DocumentationTok{\#\# Faça t = t+1}
\NormalTok{    t }\OtherTok{\textless{}{-}}\NormalTok{ t }\SpecialCharTok{+} \DecValTok{1}
\NormalTok{  \}}
  
  \FunctionTok{return}\NormalTok{(}\FunctionTok{list}\NormalTok{(}\AttributeTok{p =}\NormalTok{ p\_vec, }\AttributeTok{lambda =}\NormalTok{ lambda\_vec))}
\NormalTok{\}}
\end{Highlighting}
\end{Shaded}

\subsubsection{Estudo Monte Carlo}\label{estudo-monte-carlo}

\begin{Shaded}
\begin{Highlighting}[]
\FunctionTok{set.seed}\NormalTok{(}\DecValTok{123}\NormalTok{)}

\CommentTok{\# Parâmetros fixos}
\NormalTok{lambda\_real }\OtherTok{\textless{}{-}} \DecValTok{3}
\NormalTok{p\_valores }\OtherTok{\textless{}{-}} \FunctionTok{c}\NormalTok{(}\DecValTok{0}\NormalTok{, }\FloatTok{0.05}\NormalTok{, }\FloatTok{0.10}\NormalTok{, }\FloatTok{0.20}\NormalTok{)}
\NormalTok{n }\OtherTok{\textless{}{-}} \DecValTok{500}          \CommentTok{\# tamanho amostral}
\NormalTok{M\_iter }\OtherTok{\textless{}{-}} \DecValTok{100}     \CommentTok{\# iterações do EM}
\NormalTok{R }\OtherTok{\textless{}{-}} \DecValTok{100}          \CommentTok{\# número de réplicas Monte Carlo}

\CommentTok{\# Armazenar resultados}
\NormalTok{resultados }\OtherTok{\textless{}{-}} \FunctionTok{data.frame}\NormalTok{(}
  \AttributeTok{p\_verdadeiro =} \FunctionTok{rep}\NormalTok{(p\_valores, }\AttributeTok{each =}\NormalTok{ R),}
  \AttributeTok{p\_estimado =} \ConstantTok{NA}\NormalTok{,}
  \AttributeTok{lambda\_estimado =} \ConstantTok{NA}
\NormalTok{)}

\CommentTok{\# Loop sobre os cenários}
\NormalTok{i }\OtherTok{\textless{}{-}} \DecValTok{1}
\ControlFlowTok{for}\NormalTok{(p\_verdadeiro }\ControlFlowTok{in}\NormalTok{ p\_valores)\{}
  
  \ControlFlowTok{for}\NormalTok{(r }\ControlFlowTok{in} \DecValTok{1}\SpecialCharTok{:}\NormalTok{R)\{}
    \CommentTok{\# Gerar dados}
\NormalTok{    x }\OtherTok{\textless{}{-}} \FunctionTok{rPoisson\_zero}\NormalTok{(n, lambda\_real, p\_verdadeiro)}
    
    \CommentTok{\# Rodar EM}
\NormalTok{    ajuste }\OtherTok{\textless{}{-}} \FunctionTok{EM\_Poisson\_zinf}\NormalTok{(x, M\_iter)}
    
    \CommentTok{\# Guardar estimativas finais (último valor)}
\NormalTok{    resultados}\SpecialCharTok{$}\NormalTok{p\_estimado[i] }\OtherTok{\textless{}{-}} \FunctionTok{tail}\NormalTok{(ajuste}\SpecialCharTok{$}\NormalTok{p, }\DecValTok{1}\NormalTok{)}
\NormalTok{    resultados}\SpecialCharTok{$}\NormalTok{lambda\_estimado[i] }\OtherTok{\textless{}{-}} \FunctionTok{tail}\NormalTok{(ajuste}\SpecialCharTok{$}\NormalTok{lambda, }\DecValTok{1}\NormalTok{)}
    
\NormalTok{    i }\OtherTok{\textless{}{-}}\NormalTok{ i }\SpecialCharTok{+} \DecValTok{1}
\NormalTok{  \}}
\NormalTok{\}}
\end{Highlighting}
\end{Shaded}

\subsubsection{Resumo}\label{resumo}

\begin{verbatim}
##   p_verdadeiro media_p desvio_p media_lambda desvio_lambda
## 1         0.00   0.004    0.006        3.001         0.081
## 2         0.05   0.049    0.015        3.011         0.081
## 3         0.10   0.100    0.018        3.010         0.089
## 4         0.20   0.201    0.020        3.010         0.104
\end{verbatim}

\subsubsection{Visualização}\label{visualizauxe7uxe3o}

\pandocbounded{\includegraphics[keepaspectratio]{lista3_MCCDII_files/figure-latex/Q5_4-1.pdf}}
\pandocbounded{\includegraphics[keepaspectratio]{lista3_MCCDII_files/figure-latex/Q5_4-2.pdf}}

Os resultados mostram que o algoritmo EM recupera bem os parâmetros
verdadeiros em todos os cenários considerados.

\section{Questão 6}\label{questuxe3o-6}

Implemente o algoritmo EM para mistura de duas distribuições Normais
apresentado nos slides e avalie a estimação em 4 cenários alterando
tamanho amostral e valores do parâmetros.

\subsection{Solução}\label{soluuxe7uxe3o-8}

Considere uma amostra aleatória \(X_1, \ldots, X_n\) com função
densidade de probabilidade dada por:

\[
f(x_i|\theta) = \sum_{k=1}^{K} P(S_i = k) P(X_i = x_i | S_i = k)
= \sum_{k=1}^{K} p_k \, \mathcal{N}(x_i|\mu_k, \sigma_k^2),
\]

onde \(\mathcal{N}(x_i|\mu_k, \sigma_k^2)\) é a densidade normal com
média \(\mu_k\) e variância \(\sigma_k^2\), e

\[
\theta = (p_1, \ldots, p_K, \mu_1, \ldots, \mu_K, \sigma_1^2, \ldots, \sigma_K^2).
\]

Em particular, para \(K=2\):

\[
f(x_i|\theta) = p\,\mathcal{N}(x_i|\mu_1, \sigma_1^2) + (1-p)\,\mathcal{N}(x_i|\mu_2, \sigma_2^2).
\]

A função de verossimilhança para dados completos é:

\[
L(\theta|x,s) = \prod_{i=1}^{n}
[p\,\mathcal{N}(x_i|\mu_1,\sigma_1^2)]^{I\{s_i=1\}}
[(1-p)\,\mathcal{N}(x_i|\mu_2,\sigma_2^2)]^{I\{s_i=2\}}.
\]

A função de log-verossimilhança correspondente é:

\[
\begin{aligned}
l(\theta|x,s) &=
\sum_{i=1}^{n} I\{s_i=1\} \left[\ln p - \tfrac{1}{2}\ln 2\pi - \tfrac{1}{2}\ln \sigma_1^2 - \frac{(x_i - \mu_1)^2}{2\sigma_1^2}\right] \\
&\quad + \sum_{i=1}^{n} I\{s_i=2\} \left[\ln (1-p) - \tfrac{1}{2}\ln 2\pi - \tfrac{1}{2}\ln \sigma_2^2 - \frac{(x_i - \mu_2)^2}{2\sigma_2^2}\right].
\end{aligned}
\]

No passo E, define-se:

\[
\gamma_{i1}^{(t)} = E[I\{S_i=1\} | x_i, \theta^{(t)}] = P(S_i=1|x_i,\theta^{(t)}),
\] \[
\gamma_{i2}^{(t)} = E[I\{S_i=2\} | x_i, \theta^{(t)}] = P(S_i=2|x_i,\theta^{(t)}).
\]

Assim,

\[
\begin{aligned}
Q(\theta|\theta^{(t)}) &=
\sum_{i=1}^{n} \gamma_{i1}^{(t)} \left[\ln p - \tfrac{1}{2}\ln 2\pi - \tfrac{1}{2}\ln \sigma_1^2 - \frac{(x_i - \mu_1)^2}{2\sigma_1^2}\right] \\
&\quad + \gamma_{i2}^{(t)} \left[\ln (1-p) - \tfrac{1}{2}\ln 2\pi - \tfrac{1}{2}\ln \sigma_2^2 - \frac{(x_i - \mu_2)^2}{2\sigma_2^2}\right].
\end{aligned}
\]

No passo M, atualizamos os parâmetros maximizando
\(Q(\theta | \theta^{(t)})\):

\[
p^{(t+1)} = \frac{1}{n}\sum_{i=1}^{n}\gamma_{i1}^{(t)}, \quad
1 - p^{(t+1)} = \frac{1}{n}\sum_{i=1}^{n}\gamma_{i2}^{(t)}.
\]

De forma análoga:

\[
\mu_1^{(t+1)} = \frac{\sum_i \gamma_{i1}^{(t)} x_i}{\sum_i \gamma_{i1}^{(t)}}, \quad
\mu_2^{(t+1)} = \frac{\sum_i \gamma_{i2}^{(t)} x_i}{\sum_i \gamma_{i2}^{(t)}},
\]

\[
(\sigma_1^2)^{(t+1)} = \frac{\sum_i \gamma_{i1}^{(t)} (x_i - \mu_1^{(t+1)})^2}{\sum_i \gamma_{i1}^{(t)}}, \quad
(\sigma_2^2)^{(t+1)} = \frac{\sum_i \gamma_{i2}^{(t)} (x_i - \mu_2^{(t+1)})^2}{\sum_i \gamma_{i2}^{(t)}}.
\]

Passo a Passo do Algoritmo:

\begin{enumerate}
\def\labelenumi{\arabic{enumi}.}
\tightlist
\item
  Inicialize
  \(p^{(0)}, \mu_1^{(0)}, \mu_2^{(0)}, (\sigma_1^2)^{(0)}, (\sigma_2^2)^{(0)}\);
\item
  Faça \(t = 0\);
\item
  Repita até convergência:

  \begin{itemize}
  \tightlist
  \item
    \textbf{Passo E:} Calcule \(\gamma_{i1}^{(t)}\) e
    \(\gamma_{i2}^{(t)}\);
  \item
    \textbf{Passo M:} Atualize
    \(p^{(t+1)}, \mu_1^{(t+1)}, \mu_2^{(t+1)}, (\sigma_1^2)^{(t+1)}, (\sigma_2^2)^{(t+1)}\);
  \item
    Faça \(t \leftarrow t + 1\).
  \end{itemize}
\end{enumerate}

\begin{Shaded}
\begin{Highlighting}[]
\NormalTok{mistura\_em }\OtherTok{\textless{}{-}} \ControlFlowTok{function}\NormalTok{(x, mu\_init, sigma\_init, p\_init, }\AttributeTok{max\_iter =} \DecValTok{100}\NormalTok{, }\AttributeTok{tol =} \FloatTok{1e{-}6}\NormalTok{) \{}
\NormalTok{  n }\OtherTok{\textless{}{-}} \FunctionTok{length}\NormalTok{(x)}
\NormalTok{  mu }\OtherTok{\textless{}{-}}\NormalTok{ mu\_init     }
\NormalTok{  sigma }\OtherTok{\textless{}{-}}\NormalTok{ sigma\_init}
\NormalTok{  p }\OtherTok{\textless{}{-}}\NormalTok{ p\_init}
  
\NormalTok{  log\_veross }\OtherTok{\textless{}{-}} \FunctionTok{numeric}\NormalTok{(max\_iter)}
  
  \ControlFlowTok{for}\NormalTok{ (t }\ControlFlowTok{in} \DecValTok{1}\SpecialCharTok{:}\NormalTok{max\_iter)\{}
\NormalTok{    dcomp1 }\OtherTok{\textless{}{-}}\NormalTok{ p }\SpecialCharTok{*} \FunctionTok{dnorm}\NormalTok{(x, }\AttributeTok{mean =}\NormalTok{ mu[}\DecValTok{1}\NormalTok{], }\AttributeTok{sd =}\NormalTok{ sigma[}\DecValTok{1}\NormalTok{])}
\NormalTok{    dcomp2 }\OtherTok{\textless{}{-}}\NormalTok{ (}\DecValTok{1} \SpecialCharTok{{-}}\NormalTok{ p) }\SpecialCharTok{*} \FunctionTok{dnorm}\NormalTok{(x, }\AttributeTok{mean =}\NormalTok{ mu[}\DecValTok{2}\NormalTok{], }\AttributeTok{sd =}\NormalTok{ sigma[}\DecValTok{2}\NormalTok{])}
    
\NormalTok{    dtotal }\OtherTok{\textless{}{-}}\NormalTok{ dcomp1 }\SpecialCharTok{+}\NormalTok{ dcomp2}
  
    
\NormalTok{    gamma1 }\OtherTok{\textless{}{-}}\NormalTok{ dcomp1 }\SpecialCharTok{/}\NormalTok{ dtotal }
\NormalTok{    gamma2 }\OtherTok{\textless{}{-}}\NormalTok{ dcomp2 }\SpecialCharTok{/}\NormalTok{ dtotal}
    
\NormalTok{    log\_veross[t] }\OtherTok{\textless{}{-}} \FunctionTok{sum}\NormalTok{(}\FunctionTok{log}\NormalTok{(dtotal))}
    
\NormalTok{    n1 }\OtherTok{\textless{}{-}} \FunctionTok{sum}\NormalTok{(gamma1)}
\NormalTok{    n2 }\OtherTok{\textless{}{-}} \FunctionTok{sum}\NormalTok{(gamma2)}
    
\NormalTok{    p\_cand }\OtherTok{\textless{}{-}}\NormalTok{ n1 }\SpecialCharTok{/}\NormalTok{ n}
    
\NormalTok{    mu1\_cand }\OtherTok{\textless{}{-}} \FunctionTok{sum}\NormalTok{(gamma1 }\SpecialCharTok{*}\NormalTok{ x) }\SpecialCharTok{/}\NormalTok{ n1}
\NormalTok{    mu2\_cand }\OtherTok{\textless{}{-}} \FunctionTok{sum}\NormalTok{(gamma2 }\SpecialCharTok{*}\NormalTok{ x) }\SpecialCharTok{/}\NormalTok{ n2}
    
\NormalTok{    sigma1\_sq\_cand }\OtherTok{\textless{}{-}} \FunctionTok{sum}\NormalTok{(gamma1 }\SpecialCharTok{*}\NormalTok{ (x }\SpecialCharTok{{-}}\NormalTok{ mu1\_cand)}\SpecialCharTok{\^{}}\DecValTok{2}\NormalTok{) }\SpecialCharTok{/}\NormalTok{ n1}
\NormalTok{    sigma2\_sq\_cand }\OtherTok{\textless{}{-}} \FunctionTok{sum}\NormalTok{(gamma2 }\SpecialCharTok{*}\NormalTok{ (x }\SpecialCharTok{{-}}\NormalTok{ mu2\_cand)}\SpecialCharTok{\^{}}\DecValTok{2}\NormalTok{) }\SpecialCharTok{/}\NormalTok{ n2}

    \ControlFlowTok{if}\NormalTok{ (sigma1\_sq\_cand }\SpecialCharTok{\textless{}=} \DecValTok{0} \SpecialCharTok{||}\NormalTok{ sigma2\_sq\_cand }\SpecialCharTok{\textless{}=} \DecValTok{0}\NormalTok{) \{}
      \ControlFlowTok{break}
\NormalTok{    \}}
    
\NormalTok{    p }\OtherTok{\textless{}{-}}\NormalTok{ p\_cand}
\NormalTok{    mu }\OtherTok{\textless{}{-}} \FunctionTok{c}\NormalTok{(mu1\_cand, mu2\_cand)}
\NormalTok{    sigma }\OtherTok{\textless{}{-}} \FunctionTok{c}\NormalTok{(}\FunctionTok{sqrt}\NormalTok{(sigma1\_sq\_cand), }\FunctionTok{sqrt}\NormalTok{(sigma2\_sq\_cand))}
    
    \ControlFlowTok{if}\NormalTok{ (t }\SpecialCharTok{\textgreater{}} \DecValTok{1} \SpecialCharTok{\&\&} \FunctionTok{abs}\NormalTok{(log\_veross[t] }\SpecialCharTok{{-}}\NormalTok{ log\_veross[t}\DecValTok{{-}1}\NormalTok{]) }\SpecialCharTok{\textless{}}\NormalTok{ tol) \{}
\NormalTok{      log\_veross }\OtherTok{\textless{}{-}}\NormalTok{ log\_veross[}\DecValTok{1}\SpecialCharTok{:}\NormalTok{t] }
      \ControlFlowTok{break}
\NormalTok{    \}}
\NormalTok{  \}}
  
  \FunctionTok{return}\NormalTok{(}\FunctionTok{list}\NormalTok{(}
    \AttributeTok{p =}\NormalTok{ p,}
    \AttributeTok{mu =}\NormalTok{ mu,}
    \AttributeTok{sigma =}\NormalTok{ sigma,}
    \AttributeTok{log\_veross =}\NormalTok{ log\_veross,}
    \AttributeTok{iter =}\NormalTok{ t,}
    \AttributeTok{gammas =} \FunctionTok{data.frame}\NormalTok{(}\AttributeTok{gamma1 =}\NormalTok{ gamma1, }\AttributeTok{gamma2 =}\NormalTok{ gamma2)}
\NormalTok{  ))}
\NormalTok{\}}
\end{Highlighting}
\end{Shaded}

\subsubsection{Cenários:}\label{cenuxe1rios}

\begin{Shaded}
\begin{Highlighting}[]
\NormalTok{gerar\_mistura }\OtherTok{\textless{}{-}} \ControlFlowTok{function}\NormalTok{(n, p\_verd, mu\_verd, sigma\_verd) \{}
\NormalTok{  componente }\OtherTok{\textless{}{-}} \FunctionTok{sample}\NormalTok{(}\DecValTok{1}\SpecialCharTok{:}\DecValTok{2}\NormalTok{, }\AttributeTok{size =}\NormalTok{ n, }\AttributeTok{replace =} \ConstantTok{TRUE}\NormalTok{, }\AttributeTok{prob =} \FunctionTok{c}\NormalTok{(p\_verd, }\DecValTok{1} \SpecialCharTok{{-}}\NormalTok{ p\_verd))}
\NormalTok{  x }\OtherTok{\textless{}{-}} \FunctionTok{ifelse}\NormalTok{(componente }\SpecialCharTok{==} \DecValTok{1}\NormalTok{, }
              \FunctionTok{rnorm}\NormalTok{(n, }\AttributeTok{mean =}\NormalTok{ mu\_verd[}\DecValTok{1}\NormalTok{], }\AttributeTok{sd =}\NormalTok{ sigma\_verd[}\DecValTok{1}\NormalTok{]),}
              \FunctionTok{rnorm}\NormalTok{(n, }\AttributeTok{mean =}\NormalTok{ mu\_verd[}\DecValTok{2}\NormalTok{], }\AttributeTok{sd =}\NormalTok{ sigma\_verd[}\DecValTok{2}\NormalTok{]))}
  \FunctionTok{return}\NormalTok{(x)}
\NormalTok{\}}

\NormalTok{avaliar\_cenario }\OtherTok{\textless{}{-}} \ControlFlowTok{function}\NormalTok{(nome\_cenario, n, p\_verd, mu\_verd, sigma\_verd) \{}
  \FunctionTok{set.seed}\NormalTok{(}\DecValTok{123}\NormalTok{)}
  
\NormalTok{  x }\OtherTok{\textless{}{-}} \FunctionTok{gerar\_mistura}\NormalTok{(n, p\_verd, mu\_verd, sigma\_verd)}

\NormalTok{  mu\_init }\OtherTok{\textless{}{-}} \FunctionTok{quantile}\NormalTok{(x, }\FunctionTok{c}\NormalTok{(}\FloatTok{0.25}\NormalTok{, }\FloatTok{0.75}\NormalTok{)) }
\NormalTok{  sigma\_init }\OtherTok{\textless{}{-}} \FunctionTok{c}\NormalTok{(}\FunctionTok{sd}\NormalTok{(x), }\FunctionTok{sd}\NormalTok{(x))         }
\NormalTok{  p\_init }\OtherTok{\textless{}{-}} \FloatTok{0.5}                          
  
\NormalTok{  resultado\_em }\OtherTok{\textless{}{-}} \FunctionTok{mistura\_em}\NormalTok{(x, mu\_init, sigma\_init, p\_init)}

\NormalTok{  ordem\_verd }\OtherTok{\textless{}{-}} \FunctionTok{order}\NormalTok{(mu\_verd)}
\NormalTok{  mu\_verd }\OtherTok{\textless{}{-}}\NormalTok{ mu\_verd[ordem\_verd]}
\NormalTok{  sigma\_verd }\OtherTok{\textless{}{-}}\NormalTok{ sigma\_verd[ordem\_verd]}
\NormalTok{  p\_verd }\OtherTok{\textless{}{-}} \ControlFlowTok{if}\NormalTok{ (ordem\_verd[}\DecValTok{1}\NormalTok{] }\SpecialCharTok{==} \DecValTok{1}\NormalTok{) p\_verd }\ControlFlowTok{else}\NormalTok{ (}\DecValTok{1} \SpecialCharTok{{-}}\NormalTok{ p\_verd)}

\NormalTok{  ordem\_est }\OtherTok{\textless{}{-}} \FunctionTok{order}\NormalTok{(resultado\_em}\SpecialCharTok{$}\NormalTok{mu)}
\NormalTok{  mu\_est }\OtherTok{\textless{}{-}}\NormalTok{ resultado\_em}\SpecialCharTok{$}\NormalTok{mu[ordem\_est]}
\NormalTok{  sigma\_est }\OtherTok{\textless{}{-}}\NormalTok{ resultado\_em}\SpecialCharTok{$}\NormalTok{sigma[ordem\_est]}
\NormalTok{  p\_est }\OtherTok{\textless{}{-}} \ControlFlowTok{if}\NormalTok{ (ordem\_est[}\DecValTok{1}\NormalTok{] }\SpecialCharTok{==} \DecValTok{1}\NormalTok{) resultado\_em}\SpecialCharTok{$}\NormalTok{p }\ControlFlowTok{else}\NormalTok{ (}\DecValTok{1} \SpecialCharTok{{-}}\NormalTok{ resultado\_em}\SpecialCharTok{$}\NormalTok{p)}

\NormalTok{  tabela\_resultados }\OtherTok{\textless{}{-}} \FunctionTok{data.frame}\NormalTok{(}
    \AttributeTok{Parametros =} \FunctionTok{c}\NormalTok{(}\StringTok{"p"}\NormalTok{, }\StringTok{"Mu"}\NormalTok{, }\StringTok{"Sigma"}\NormalTok{),}
    \AttributeTok{Verdadeiro =} \FunctionTok{c}\NormalTok{(}
      \FunctionTok{sprintf}\NormalTok{(}\StringTok{"\%.3f"}\NormalTok{, p\_verd),}
      \FunctionTok{sprintf}\NormalTok{(}\StringTok{"(\%.3f, \%.3f)"}\NormalTok{, mu\_verd[}\DecValTok{1}\NormalTok{], mu\_verd[}\DecValTok{2}\NormalTok{]),}
      \FunctionTok{sprintf}\NormalTok{(}\StringTok{"(\%.3f, \%.3f)"}\NormalTok{, sigma\_verd[}\DecValTok{1}\NormalTok{], sigma\_verd[}\DecValTok{2}\NormalTok{])}
\NormalTok{    ),}
    \AttributeTok{Estimado\_EM =} \FunctionTok{c}\NormalTok{(}
      \FunctionTok{sprintf}\NormalTok{(}\StringTok{"\%.3f"}\NormalTok{, p\_est),}
      \FunctionTok{sprintf}\NormalTok{(}\StringTok{"(\%.3f, \%.3f)"}\NormalTok{, mu\_est[}\DecValTok{1}\NormalTok{], mu\_est[}\DecValTok{2}\NormalTok{]),}
      \FunctionTok{sprintf}\NormalTok{(}\StringTok{"(\%.3f, \%.3f)"}\NormalTok{, sigma\_est[}\DecValTok{1}\NormalTok{], sigma\_est[}\DecValTok{2}\NormalTok{])}
\NormalTok{    )}
\NormalTok{  )}
  
\NormalTok{  tabela\_formatada }\OtherTok{\textless{}{-}} \FunctionTok{kable}\NormalTok{(tabela\_resultados,}
                            \AttributeTok{booktabs =}\NormalTok{ T) }\SpecialCharTok{\%\textgreater{}\%}
    \FunctionTok{kable\_styling}\NormalTok{(}\AttributeTok{bootstrap\_options =} \FunctionTok{c}\NormalTok{(}\StringTok{"simple"}\NormalTok{, }\StringTok{"hover"}\NormalTok{), }
                  \AttributeTok{full\_width =}\NormalTok{ F,}
                  \AttributeTok{position =} \StringTok{"center"}\NormalTok{)}
 
\NormalTok{  densidade\_mistura }\OtherTok{\textless{}{-}} \ControlFlowTok{function}\NormalTok{(x, p, mu, sigma) \{}
\NormalTok{    p }\SpecialCharTok{*} \FunctionTok{dnorm}\NormalTok{(x, mu[}\DecValTok{1}\NormalTok{], sigma[}\DecValTok{1}\NormalTok{]) }\SpecialCharTok{+}\NormalTok{ (}\DecValTok{1} \SpecialCharTok{{-}}\NormalTok{ p) }\SpecialCharTok{*} \FunctionTok{dnorm}\NormalTok{(x, mu[}\DecValTok{2}\NormalTok{], sigma[}\DecValTok{2}\NormalTok{])}
\NormalTok{  \}}
  
\NormalTok{  df\_plot }\OtherTok{\textless{}{-}} \FunctionTok{data.frame}\NormalTok{(}\AttributeTok{x =}\NormalTok{ x)}
  
\NormalTok{  p }\OtherTok{\textless{}{-}} \FunctionTok{ggplot}\NormalTok{(df\_plot, }\FunctionTok{aes}\NormalTok{(}\AttributeTok{x =}\NormalTok{ x)) }\SpecialCharTok{+}
    \FunctionTok{geom\_histogram}\NormalTok{(}\FunctionTok{aes}\NormalTok{(}\AttributeTok{y =}\NormalTok{ ..density..), }\AttributeTok{bins =} \DecValTok{50}\NormalTok{, }\AttributeTok{fill =} \StringTok{"lightblue"}\NormalTok{, }
                   \AttributeTok{color =} \StringTok{"black"}\NormalTok{, }\AttributeTok{alpha =} \FloatTok{0.7}\NormalTok{) }\SpecialCharTok{+}
    \FunctionTok{stat\_function}\NormalTok{(}\AttributeTok{fun =}\NormalTok{ densidade\_mistura, }\AttributeTok{args =} \FunctionTok{list}\NormalTok{(}\AttributeTok{p =}\NormalTok{ p\_verd, }\AttributeTok{mu =}\NormalTok{ mu\_verd, }
                                                       \AttributeTok{sigma =}\NormalTok{ sigma\_verd),}
                  \FunctionTok{aes}\NormalTok{(}\AttributeTok{color =} \StringTok{"Verdadeira"}\NormalTok{), }\AttributeTok{size =} \FloatTok{1.2}\NormalTok{) }\SpecialCharTok{+}
    \FunctionTok{stat\_function}\NormalTok{(}\AttributeTok{fun =}\NormalTok{ densidade\_mistura, }\AttributeTok{args =} \FunctionTok{list}\NormalTok{(}\AttributeTok{p =}\NormalTok{ p\_est, }\AttributeTok{mu =}\NormalTok{ mu\_est, }
                                                       \AttributeTok{sigma =}\NormalTok{ sigma\_est),}
                  \FunctionTok{aes}\NormalTok{(}\AttributeTok{color =} \StringTok{"Estimada"}\NormalTok{), }\AttributeTok{size =} \FloatTok{1.2}\NormalTok{, }\AttributeTok{linetype =} \StringTok{"dashed"}\NormalTok{) }\SpecialCharTok{+}
    \FunctionTok{scale\_color\_manual}\NormalTok{(}\AttributeTok{name =} \StringTok{"Densidade"}\NormalTok{, }
                       \AttributeTok{values =} \FunctionTok{c}\NormalTok{(}\StringTok{"Verdadeira"} \OtherTok{=} \StringTok{"red3"}\NormalTok{, }\StringTok{"Estimada"} \OtherTok{=} \StringTok{"dodgerblue4"}\NormalTok{)) }\SpecialCharTok{+}
    \FunctionTok{labs}\NormalTok{(}\AttributeTok{title =}\NormalTok{ nome\_cenario,}
         \AttributeTok{subtitle =} \FunctionTok{paste}\NormalTok{(}\StringTok{"n ="}\NormalTok{, n),}
         \AttributeTok{x =} \StringTok{"Valor"}\NormalTok{,}
         \AttributeTok{y =} \StringTok{"Densidade"}\NormalTok{) }\SpecialCharTok{+}
    \FunctionTok{theme\_minimal}\NormalTok{()}
  
  \FunctionTok{print}\NormalTok{(p)}
  
  \FunctionTok{cat}\NormalTok{(}\FunctionTok{paste}\NormalTok{(}\StringTok{"O algoritmo convergiu em"}\NormalTok{, resultado\_em}\SpecialCharTok{$}\NormalTok{iter, }\StringTok{"iterações"}\NormalTok{))}
  
  \FunctionTok{return}\NormalTok{(tabela\_formatada)}
\NormalTok{\}}
\end{Highlighting}
\end{Shaded}

\textbf{Cenário 1}

\begin{itemize}
\item
  \(n = 2000\)
\item
  \(p = 0.5\)
\item
  \(\mu = (0, 5)\)
\item
  \(\sigma = (1, 1)\)
\end{itemize}

\pandocbounded{\includegraphics[keepaspectratio]{lista3_MCCDII_files/figure-latex/mistura-normais-cenario1-1.pdf}}

\begin{verbatim}
## O algoritmo convergiu em 13 iterações
\end{verbatim}

\begin{longtable}[t]{lll}
\toprule
Parametros & Verdadeiro & Estimado\_EM\\
\midrule
p & 0.500 & 0.497\\
Mu & (0.000, 5.000) & (0.010, 5.003)\\
Sigma & (1.000, 1.000) & (0.995, 0.974)\\
\bottomrule
\end{longtable}

\textbf{Cenário 2}

\begin{itemize}
\item
  \(n = 2000\)
\item
  \(p = 0.5\)
\item
  \(\mu = (0, 2)\)
\item
  \(\sigma = (1, 0.8)\)
\end{itemize}

\pandocbounded{\includegraphics[keepaspectratio]{lista3_MCCDII_files/figure-latex/mistura-normais-cenario2-1.pdf}}

\begin{verbatim}
## O algoritmo convergiu em 100 iterações
\end{verbatim}

\begin{longtable}[t]{lll}
\toprule
Parametros & Verdadeiro & Estimado\_EM\\
\midrule
p & 0.500 & 0.498\\
Mu & (0.000, 2.000) & (0.006, 2.000)\\
Sigma & (1.000, 0.800) & (0.993, 0.784)\\
\bottomrule
\end{longtable}

\textbf{Cenário 3}

\begin{itemize}
\item
  \(n = 200\)
\item
  \(p = 0.5\)
\item
  \(\mu = (0, 6)\)
\item
  \(\sigma = (1, 1)\)
\end{itemize}

\pandocbounded{\includegraphics[keepaspectratio]{lista3_MCCDII_files/figure-latex/mistura-normais-cenario3-1.pdf}}

\begin{verbatim}
## O algoritmo convergiu em 15 iterações
\end{verbatim}

\begin{longtable}[t]{lll}
\toprule
Parametros & Verdadeiro & Estimado\_EM\\
\midrule
p & 0.500 & 0.491\\
Mu & (0.000, 5.000) & (-0.009, 5.159)\\
Sigma & (1.000, 1.000) & (0.899, 0.982)\\
\bottomrule
\end{longtable}

\textbf{Cenário 4}

\begin{itemize}
\item
  \(n = 200\)
\item
  \(p = 0.5\)
\item
  \(\mu = (0, 2)\)
\item
  \(\sigma = (1, 0.8)\)
\end{itemize}

\pandocbounded{\includegraphics[keepaspectratio]{lista3_MCCDII_files/figure-latex/mistura-normais-cenario4-1.pdf}}

\begin{verbatim}
## O algoritmo convergiu em 53 iterações
\end{verbatim}

\begin{longtable}[t]{lll}
\toprule
Parametros & Verdadeiro & Estimado\_EM\\
\midrule
p & 0.500 & 0.466\\
Mu & (0.000, 2.000) & (-0.175, 2.144)\\
Sigma & (1.000, 0.800) & (0.705, 0.717)\\
\bottomrule
\end{longtable}

\textbf{Cenário 5}

\begin{itemize}
\item
  \(n = 700\)
\item
  \(p = 0.3\)
\item
  \(\mu = (0, 5)\)
\item
  \(\sigma = (1, 1)\)
\end{itemize}

\pandocbounded{\includegraphics[keepaspectratio]{lista3_MCCDII_files/figure-latex/mistura-normais-cenario5-1.pdf}}

\begin{verbatim}
## O algoritmo convergiu em 22 iterações
\end{verbatim}

\begin{longtable}[t]{lll}
\toprule
Parametros & Verdadeiro & Estimado\_EM\\
\midrule
p & 0.300 & 0.299\\
Mu & (0.000, 5.000) & (-0.042, 5.022)\\
Sigma & (1.000, 1.000) & (1.024, 0.999)\\
\bottomrule
\end{longtable}

\textbf{Cenário 6}

\begin{itemize}
\item
  \(n = 700\)
\item
  \(p = 0.3\)
\item
  \(\mu = (0, 2)\)
\item
  \(\sigma = (1, 0.8)\)
\end{itemize}

\pandocbounded{\includegraphics[keepaspectratio]{lista3_MCCDII_files/figure-latex/mistura-normais-cenario6-1.pdf}}

\begin{verbatim}
## O algoritmo convergiu em 100 iterações
\end{verbatim}

\begin{longtable}[t]{lll}
\toprule
Parametros & Verdadeiro & Estimado\_EM\\
\midrule
p & 0.300 & 0.556\\
Mu & (0.000, 2.000) & (0.752, 2.230)\\
Sigma & (1.000, 0.800) & (1.297, 0.625)\\
\bottomrule
\end{longtable}

\section{Questão 7}\label{questuxe3o-7}

Implemente o algoritmo ECM para mistura de duas distribuições Gamas
apresentado nos slides de variantes do algoritmo EM. Avalie as
estimativas gerando dados de alguns cenários com mistura de duas gamas.

\subsection{Solução}\label{soluuxe7uxe3o-9}

\subsubsection{Passo a passo}\label{passo-a-passo-1}

\begin{enumerate}
\def\labelenumi{\arabic{enumi}.}
\tightlist
\item
  Encontre um valor para \(\alpha\) que satisfaça uma equação envolvendo
  a função \texttt{digama}. Primeiro, tente resolver usando um método de
  busca (\texttt{uniroot}) e, se essa busca falhar, aplique
  Newton-Raphson, um método iterativo que ajusta o \(\alpha\) usando a
  derivada da função \texttt{digama} (\texttt{trigama}), garantindo que
  o valor final seja positivo.
\end{enumerate}

\begin{itemize}
\tightlist
\item
  A função do ECM deve receber os dados observados e os parâmetros
  iniciais para a mistura. Depois, é necessário ter uma estrutura para
  registrar a evolução dos parâmetros ao longo das iterações e definir
  um loop que irá atualizar os parâmetros até atingir convergência ou o
  número máximo de iterações.
\end{itemize}

\begin{enumerate}
\def\labelenumi{\arabic{enumi}.}
\setcounter{enumi}{1}
\item
  Passo E: calcule a probabilidade de cada observação pertencer à
  primeira ou à segunda componente da mistura usando as densidades
  atuais das duas Gamas. Não é obrigatório, mas seria uma boa prática
  limitar essas probabilidades para evitar valores extremos ou divisões
  por zero. Compute somas ponderadas das observações e das
  probabilidades para usar na atualização dos parâmetros.
\item
  Passo CM: primeiro, atualize a proporção da primeira componente
  (\(\delta\)) como a média das probabilidades calculadas no passo E. Em
  segundo lugar, atualize os parâmetros de taxa (\(\beta\)) para cada
  componente usando a média ponderada das observações, e finalmente,
  calcule os valores intermediários que servem de entrada para resolver
  numericamente os parâmetros de forma (\(\alpha\)) usando a função
  criada no passo 1.
\item
  Estabeleça um critério de convergência (mudança menor que a
  tolerância) e registre os resultados assim que convergir.
\item
  Gere os dados da mistura de das Gamas para os diferentes cenários a
  partir de uma variável indicadora (define a qual componente cada
  observação pertence).
\item
  Analise o número de iterações e os valores das estimativas dos
  parâmetros.
\end{enumerate}

\subsubsection{Implementação do
código}\label{implementauxe7uxe3o-do-cuxf3digo-1}

\begin{Shaded}
\begin{Highlighting}[]
\CommentTok{\# solução numérica para alfa}
\NormalTok{resolver\_alfa }\OtherTok{\textless{}{-}} \ControlFlowTok{function}\NormalTok{(C\_k, }\AttributeTok{alfa\_inicial =} \DecValTok{1}\NormalTok{) \{}
\NormalTok{  fun }\OtherTok{\textless{}{-}} \ControlFlowTok{function}\NormalTok{(a) }\FunctionTok{digamma}\NormalTok{(a) }\SpecialCharTok{{-}}\NormalTok{ C\_k}
\NormalTok{  resultado }\OtherTok{\textless{}{-}} \FunctionTok{tryCatch}\NormalTok{(\{}
    \FunctionTok{uniroot}\NormalTok{(fun, }\AttributeTok{interval =} \FunctionTok{c}\NormalTok{(}\FloatTok{1e{-}8}\NormalTok{, }\FloatTok{1e6}\NormalTok{), }\AttributeTok{tol =} \FloatTok{1e{-}8}\NormalTok{)}\SpecialCharTok{$}\NormalTok{root}
\NormalTok{  \}, }\AttributeTok{error =} \ControlFlowTok{function}\NormalTok{(e) \{}
    \CommentTok{\# newton{-}raphson}
\NormalTok{    a }\OtherTok{\textless{}{-}} \FunctionTok{max}\NormalTok{(alfa\_inicial, }\FloatTok{1e{-}6}\NormalTok{)}
    \ControlFlowTok{for}\NormalTok{ (i }\ControlFlowTok{in} \DecValTok{1}\SpecialCharTok{:}\DecValTok{100}\NormalTok{) \{}
\NormalTok{      g }\OtherTok{\textless{}{-}} \FunctionTok{digamma}\NormalTok{(a) }\SpecialCharTok{{-}}\NormalTok{ C\_k}
\NormalTok{      tg }\OtherTok{\textless{}{-}} \FunctionTok{trigamma}\NormalTok{(a)}
      \ControlFlowTok{if}\NormalTok{ (}\FunctionTok{is.na}\NormalTok{(g) }\SpecialCharTok{||} \FunctionTok{is.na}\NormalTok{(tg) }\SpecialCharTok{||}\NormalTok{ tg }\SpecialCharTok{\textless{}=} \DecValTok{0}\NormalTok{) }\ControlFlowTok{break}
\NormalTok{      a\_novo }\OtherTok{\textless{}{-}}\NormalTok{ a }\SpecialCharTok{{-}}\NormalTok{ g }\SpecialCharTok{/}\NormalTok{ tg}
      \ControlFlowTok{if}\NormalTok{ (a\_novo }\SpecialCharTok{\textless{}=} \DecValTok{0}\NormalTok{) a\_novo }\OtherTok{\textless{}{-}}\NormalTok{ a }\SpecialCharTok{/} \DecValTok{2}
      \ControlFlowTok{if}\NormalTok{ (}\FunctionTok{abs}\NormalTok{(a\_novo }\SpecialCharTok{{-}}\NormalTok{ a) }\SpecialCharTok{\textless{}} \FloatTok{1e{-}8}\NormalTok{) \{ a }\OtherTok{\textless{}{-}}\NormalTok{ a\_novo; }\ControlFlowTok{break}\NormalTok{ \}}
\NormalTok{      a }\OtherTok{\textless{}{-}}\NormalTok{ a\_novo}
\NormalTok{    \}}
\NormalTok{    a}
\NormalTok{  \})}
  \FunctionTok{as.numeric}\NormalTok{(resultado)}
\NormalTok{\}}


\NormalTok{ecm\_gama }\OtherTok{\textless{}{-}} \ControlFlowTok{function}\NormalTok{(X, parametros\_iniciais, }\AttributeTok{max\_iter =} \DecValTok{500}\NormalTok{, }\AttributeTok{tol =} \FloatTok{1e{-}6}\NormalTok{) \{}
  
  \CommentTok{\# parâmetros iniciais}
\NormalTok{  theta }\OtherTok{\textless{}{-}} \FunctionTok{as.numeric}\NormalTok{(parametros\_iniciais[}\DecValTok{1}\SpecialCharTok{:}\DecValTok{5}\NormalTok{])}
  \FunctionTok{names}\NormalTok{(theta) }\OtherTok{\textless{}{-}} \FunctionTok{c}\NormalTok{(}\StringTok{"delta"}\NormalTok{, }\StringTok{"alfa1"}\NormalTok{, }\StringTok{"alfa2"}\NormalTok{, }\StringTok{"beta1"}\NormalTok{, }\StringTok{"beta2"}\NormalTok{)}
\NormalTok{  n }\OtherTok{\textless{}{-}} \FunctionTok{length}\NormalTok{(X)}
\NormalTok{  historico }\OtherTok{\textless{}{-}} \FunctionTok{matrix}\NormalTok{(}\ConstantTok{NA}\NormalTok{, }\AttributeTok{nrow =}\NormalTok{ max\_iter, }\AttributeTok{ncol =} \DecValTok{5}\NormalTok{)}
  \FunctionTok{colnames}\NormalTok{(historico) }\OtherTok{\textless{}{-}} \FunctionTok{names}\NormalTok{(theta)}
  
  \ControlFlowTok{for}\NormalTok{ (t }\ControlFlowTok{in} \DecValTok{1}\SpecialCharTok{:}\NormalTok{max\_iter) \{}
\NormalTok{    delta\_t  }\OtherTok{\textless{}{-}}\NormalTok{ theta[}\StringTok{"delta"}\NormalTok{]}
\NormalTok{    alfa1\_t }\OtherTok{\textless{}{-}}\NormalTok{ theta[}\StringTok{"alfa1"}\NormalTok{]}
\NormalTok{    alfa2\_t }\OtherTok{\textless{}{-}}\NormalTok{ theta[}\StringTok{"alfa2"}\NormalTok{]}
\NormalTok{    beta1\_t  }\OtherTok{\textless{}{-}}\NormalTok{ theta[}\StringTok{"beta1"}\NormalTok{]}
\NormalTok{    beta2\_t  }\OtherTok{\textless{}{-}}\NormalTok{ theta[}\StringTok{"beta2"}\NormalTok{]}
    
    \CommentTok{\# passo E}
\NormalTok{    f1 }\OtherTok{\textless{}{-}} \FunctionTok{dgamma}\NormalTok{(X, }\AttributeTok{shape =}\NormalTok{ alfa1\_t, }\AttributeTok{rate =}\NormalTok{ beta1\_t)}
\NormalTok{    f2 }\OtherTok{\textless{}{-}} \FunctionTok{dgamma}\NormalTok{(X, }\AttributeTok{shape =}\NormalTok{ alfa2\_t, }\AttributeTok{rate =}\NormalTok{ beta2\_t)}
    \CommentTok{\# p\_i = P(Z\_i = 1 | x\_i, theta\^{}(t)), onde Z\_i é a variável latente com distribuição Bernoulli}
\NormalTok{    p\_i }\OtherTok{\textless{}{-}}\NormalTok{ (delta\_t }\SpecialCharTok{*}\NormalTok{ f1) }\SpecialCharTok{/}\NormalTok{ (delta\_t }\SpecialCharTok{*}\NormalTok{ f1 }\SpecialCharTok{+}\NormalTok{ (}\DecValTok{1} \SpecialCharTok{{-}}\NormalTok{ delta\_t) }\SpecialCharTok{*}\NormalTok{ f2)}
    \CommentTok{\# para garantir estabilidade numérica}
\NormalTok{    p\_i }\OtherTok{\textless{}{-}} \FunctionTok{pmin}\NormalTok{(}\FunctionTok{pmax}\NormalTok{(p\_i, }\FloatTok{1e{-}10}\NormalTok{), }\DecValTok{1} \SpecialCharTok{{-}} \FloatTok{1e{-}10}\NormalTok{)}
    
    \CommentTok{\# passo CM1: maximização condicional em relação a delta}
\NormalTok{    delta\_novo }\OtherTok{\textless{}{-}} \FunctionTok{mean}\NormalTok{(p\_i)}
    
    \CommentTok{\# passo CM2: maximização condicional em relação a alfa1 e beta1}
\NormalTok{    somatorio\_p }\OtherTok{\textless{}{-}} \FunctionTok{sum}\NormalTok{(p\_i)}
\NormalTok{    somatorio\_x\_p }\OtherTok{\textless{}{-}} \FunctionTok{sum}\NormalTok{(p\_i }\SpecialCharTok{*}\NormalTok{ X)}
    \ControlFlowTok{if}\NormalTok{ (somatorio\_p }\SpecialCharTok{\textgreater{}} \DecValTok{0}\NormalTok{) \{}
\NormalTok{      beta1\_novo }\OtherTok{\textless{}{-}}\NormalTok{ alfa1\_t }\SpecialCharTok{*}\NormalTok{ somatorio\_p }\SpecialCharTok{/}\NormalTok{ somatorio\_x\_p}
\NormalTok{      C1 }\OtherTok{\textless{}{-}}\NormalTok{ (}\FunctionTok{sum}\NormalTok{(p\_i }\SpecialCharTok{*} \FunctionTok{log}\NormalTok{(X)) }\SpecialCharTok{/}\NormalTok{ somatorio\_p) }\SpecialCharTok{+} \FunctionTok{log}\NormalTok{(beta1\_novo)}
\NormalTok{      alfa1\_novo }\OtherTok{\textless{}{-}} \FunctionTok{resolver\_alfa}\NormalTok{(C1, alfa1\_t)}
\NormalTok{    \} }\ControlFlowTok{else}\NormalTok{ \{}
\NormalTok{      beta1\_novo }\OtherTok{\textless{}{-}}\NormalTok{ beta1\_t}
\NormalTok{      alfa1\_novo }\OtherTok{\textless{}{-}}\NormalTok{ alfa1\_t}
\NormalTok{    \}}
    
    \CommentTok{\# passo CM3: maximização condicional em relação a alfa2 e beta2}
\NormalTok{    somatorio\_1mp }\OtherTok{\textless{}{-}}\NormalTok{ n }\SpecialCharTok{{-}}\NormalTok{ somatorio\_p}
\NormalTok{    somatorio\_x\_1mp }\OtherTok{\textless{}{-}} \FunctionTok{sum}\NormalTok{(X) }\SpecialCharTok{{-}}\NormalTok{ somatorio\_x\_p}
    \ControlFlowTok{if}\NormalTok{ (somatorio\_1mp }\SpecialCharTok{\textgreater{}} \DecValTok{0}\NormalTok{) \{}
\NormalTok{      beta2\_novo }\OtherTok{\textless{}{-}}\NormalTok{ alfa2\_t }\SpecialCharTok{*}\NormalTok{ somatorio\_1mp }\SpecialCharTok{/}\NormalTok{ somatorio\_x\_1mp}
\NormalTok{      C2 }\OtherTok{\textless{}{-}}\NormalTok{ (}\FunctionTok{sum}\NormalTok{((}\DecValTok{1} \SpecialCharTok{{-}}\NormalTok{ p\_i) }\SpecialCharTok{*} \FunctionTok{log}\NormalTok{(X)) }\SpecialCharTok{/}\NormalTok{ somatorio\_1mp) }\SpecialCharTok{+} \FunctionTok{log}\NormalTok{(beta2\_novo)}
\NormalTok{      alfa2\_novo }\OtherTok{\textless{}{-}} \FunctionTok{resolver\_alfa}\NormalTok{(C2, alfa2\_t)}
\NormalTok{    \} }\ControlFlowTok{else}\NormalTok{ \{}
\NormalTok{      beta2\_novo }\OtherTok{\textless{}{-}}\NormalTok{ beta2\_t}
\NormalTok{      alfa2\_novo }\OtherTok{\textless{}{-}}\NormalTok{ alfa2\_t}
\NormalTok{    \}}
    
    \CommentTok{\# atualização dos parâmetros}
\NormalTok{    theta\_novo }\OtherTok{\textless{}{-}} \FunctionTok{c}\NormalTok{(delta\_novo, alfa1\_novo, alfa2\_novo, beta1\_novo, beta2\_novo)}
    \FunctionTok{names}\NormalTok{(theta\_novo) }\OtherTok{\textless{}{-}} \FunctionTok{names}\NormalTok{(theta)}
    
    \CommentTok{\# critério de convergência}
\NormalTok{    crit }\OtherTok{\textless{}{-}} \FunctionTok{max}\NormalTok{(}\FunctionTok{abs}\NormalTok{(theta\_novo }\SpecialCharTok{{-}}\NormalTok{ theta) }\SpecialCharTok{/} \FunctionTok{pmax}\NormalTok{(}\FunctionTok{abs}\NormalTok{(theta), }\FloatTok{1e{-}8}\NormalTok{))}
\NormalTok{    theta }\OtherTok{\textless{}{-}}\NormalTok{ theta\_novo}
\NormalTok{    historico[t, ] }\OtherTok{\textless{}{-}}\NormalTok{ theta}
    
    \ControlFlowTok{if}\NormalTok{ (crit }\SpecialCharTok{\textless{}}\NormalTok{ tol) }\ControlFlowTok{break}
    
\NormalTok{  \}}
  
  \FunctionTok{list}\NormalTok{(}\AttributeTok{estimativas =}\NormalTok{ theta, }\AttributeTok{iteracoes =}\NormalTok{ t, }\AttributeTok{historico =}\NormalTok{ historico[}\DecValTok{1}\SpecialCharTok{:}\NormalTok{t, ])}
\NormalTok{\}}

\CommentTok{\# gerando os dados}
\NormalTok{mistura\_gama }\OtherTok{\textless{}{-}} \ControlFlowTok{function}\NormalTok{(n, delta, alfa1, alfa2, beta1, beta2) \{}
\NormalTok{  Z }\OtherTok{\textless{}{-}} \FunctionTok{rbinom}\NormalTok{(n, }\DecValTok{1}\NormalTok{, delta)}
  \FunctionTok{rgamma}\NormalTok{(n, }\AttributeTok{shape =} \FunctionTok{ifelse}\NormalTok{(Z }\SpecialCharTok{==} \DecValTok{1}\NormalTok{, alfa1, alfa2),}
         \AttributeTok{rate =} \FunctionTok{ifelse}\NormalTok{(Z }\SpecialCharTok{==} \DecValTok{1}\NormalTok{, beta1, beta2))}
\NormalTok{\}}

\FunctionTok{set.seed}\NormalTok{(}\DecValTok{123456789}\NormalTok{)}

\CommentTok{\# cenário 1: mistura bem separada}
\NormalTok{parametros1 }\OtherTok{\textless{}{-}} \FunctionTok{c}\NormalTok{(}\AttributeTok{delta =} \FloatTok{0.6}\NormalTok{, }\AttributeTok{alfa1 =} \DecValTok{5}\NormalTok{, }\AttributeTok{alfa2 =} \DecValTok{2}\NormalTok{, }\AttributeTok{beta1 =} \DecValTok{1}\NormalTok{, }\AttributeTok{beta2 =} \FloatTok{0.2}\NormalTok{)}
\NormalTok{dados1 }\OtherTok{\textless{}{-}} \FunctionTok{mistura\_gama}\NormalTok{(}\DecValTok{500}\NormalTok{, parametros1[}\StringTok{"delta"}\NormalTok{], parametros1[}\StringTok{"alfa1"}\NormalTok{],}
\NormalTok{                        parametros1[}\StringTok{"alfa2"}\NormalTok{], parametros1[}\StringTok{"beta1"}\NormalTok{], parametros1[}\StringTok{"beta2"}\NormalTok{])}
\NormalTok{resultado1 }\OtherTok{\textless{}{-}} \FunctionTok{ecm\_gama}\NormalTok{(dados1, }\FunctionTok{c}\NormalTok{(}\FloatTok{0.5}\NormalTok{, }\DecValTok{1}\NormalTok{, }\DecValTok{1}\NormalTok{, }\FloatTok{0.5}\NormalTok{, }\FloatTok{0.5}\NormalTok{))}

\CommentTok{\# cenário 2: mistura com maior influência da segunda distribuição}
\NormalTok{parametros2 }\OtherTok{\textless{}{-}} \FunctionTok{c}\NormalTok{(}\AttributeTok{delta =} \FloatTok{0.2}\NormalTok{, }\AttributeTok{alfa1 =} \DecValTok{10}\NormalTok{, }\AttributeTok{alfa2 =} \DecValTok{5}\NormalTok{, }\AttributeTok{beta1 =} \DecValTok{2}\NormalTok{, }\AttributeTok{beta2 =} \FloatTok{0.5}\NormalTok{)}
\NormalTok{dados2 }\OtherTok{\textless{}{-}} \FunctionTok{mistura\_gama}\NormalTok{(}\DecValTok{1000}\NormalTok{, parametros2[}\StringTok{"delta"}\NormalTok{], parametros2[}\StringTok{"alfa1"}\NormalTok{],}
\NormalTok{                        parametros2[}\StringTok{"alfa2"}\NormalTok{], parametros2[}\StringTok{"beta1"}\NormalTok{], parametros2[}\StringTok{"beta2"}\NormalTok{])}
\NormalTok{resultado2 }\OtherTok{\textless{}{-}} \FunctionTok{ecm\_gama}\NormalTok{(dados2, }\FunctionTok{c}\NormalTok{(}\FloatTok{0.5}\NormalTok{, }\DecValTok{1}\NormalTok{, }\DecValTok{1}\NormalTok{, }\FloatTok{0.5}\NormalTok{, }\FloatTok{0.5}\NormalTok{))}

\CommentTok{\# cenário 3: mistura com densidades sobrepostas}

\NormalTok{parametros3 }\OtherTok{\textless{}{-}} \FunctionTok{c}\NormalTok{(}\AttributeTok{delta =} \FloatTok{0.5}\NormalTok{, }\AttributeTok{alfa1 =} \DecValTok{2}\NormalTok{, }\AttributeTok{alfa2 =} \DecValTok{3}\NormalTok{, }\AttributeTok{beta1 =} \DecValTok{1}\NormalTok{, }\AttributeTok{beta2 =} \FloatTok{1.2}\NormalTok{)}
\NormalTok{dados3 }\OtherTok{\textless{}{-}} \FunctionTok{mistura\_gama}\NormalTok{(}\DecValTok{800}\NormalTok{, parametros3[}\StringTok{"delta"}\NormalTok{], parametros3[}\StringTok{"alfa1"}\NormalTok{],}
\NormalTok{                        parametros3[}\StringTok{"alfa2"}\NormalTok{], parametros3[}\StringTok{"beta1"}\NormalTok{], parametros3[}\StringTok{"beta2"}\NormalTok{])}
\NormalTok{resultado3 }\OtherTok{\textless{}{-}} \FunctionTok{ecm\_gama}\NormalTok{(dados3, }\FunctionTok{c}\NormalTok{(}\FloatTok{0.5}\NormalTok{, }\DecValTok{1}\NormalTok{, }\DecValTok{1}\NormalTok{, }\FloatTok{0.5}\NormalTok{, }\FloatTok{0.5}\NormalTok{))}

\NormalTok{resultados }\OtherTok{\textless{}{-}} \FunctionTok{data.frame}\NormalTok{(}
\NormalTok{  Cenário }\OtherTok{=} \FunctionTok{c}\NormalTok{(}\StringTok{"Cenário 1: Separada"}\NormalTok{, }\StringTok{"Cenário 2: Desigual"}\NormalTok{, }\StringTok{"Cenário 3: Sobreposta"}\NormalTok{),}
\NormalTok{  Iterações }\OtherTok{=} \FunctionTok{c}\NormalTok{(resultado1}\SpecialCharTok{$}\NormalTok{iteracoes, resultado2}\SpecialCharTok{$}\NormalTok{iteracoes, resultado3}\SpecialCharTok{$}\NormalTok{iteracoes),}
  \FunctionTok{round}\NormalTok{(}\FunctionTok{rbind}\NormalTok{(resultado1}\SpecialCharTok{$}\NormalTok{estimativas, resultado2}\SpecialCharTok{$}\NormalTok{estimativas, resultado3}\SpecialCharTok{$}\NormalTok{estimativas), }\DecValTok{4}\NormalTok{)}
\NormalTok{)}
\FunctionTok{kable}\NormalTok{(resultados)}
\end{Highlighting}
\end{Shaded}

\begin{longtable}[]{@{}
  >{\raggedright\arraybackslash}p{(\linewidth - 12\tabcolsep) * \real{0.3333}}
  >{\raggedleft\arraybackslash}p{(\linewidth - 12\tabcolsep) * \real{0.1515}}
  >{\raggedleft\arraybackslash}p{(\linewidth - 12\tabcolsep) * \real{0.0909}}
  >{\raggedleft\arraybackslash}p{(\linewidth - 12\tabcolsep) * \real{0.1061}}
  >{\raggedleft\arraybackslash}p{(\linewidth - 12\tabcolsep) * \real{0.1061}}
  >{\raggedleft\arraybackslash}p{(\linewidth - 12\tabcolsep) * \real{0.1061}}
  >{\raggedleft\arraybackslash}p{(\linewidth - 12\tabcolsep) * \real{0.1061}}@{}}
\toprule\noalign{}
\begin{minipage}[b]{\linewidth}\raggedright
Cenário
\end{minipage} & \begin{minipage}[b]{\linewidth}\raggedleft
Iterações
\end{minipage} & \begin{minipage}[b]{\linewidth}\raggedleft
delta
\end{minipage} & \begin{minipage}[b]{\linewidth}\raggedleft
alfa1
\end{minipage} & \begin{minipage}[b]{\linewidth}\raggedleft
alfa2
\end{minipage} & \begin{minipage}[b]{\linewidth}\raggedleft
beta1
\end{minipage} & \begin{minipage}[b]{\linewidth}\raggedleft
beta2
\end{minipage} \\
\midrule\noalign{}
\endhead
\bottomrule\noalign{}
\endlastfoot
Cenário 1: Separada & 55 & 0.5 & 2.2790 & 2.2790 & 0.3181 & 0.3181 \\
Cenário 2: Desigual & 93 & 0.5 & 4.0214 & 4.0214 & 0.4477 & 0.4477 \\
Cenário 3: Sobreposta & 57 & 0.5 & 2.3769 & 2.3769 & 1.0799 & 1.0799 \\
\end{longtable}

\begin{Shaded}
\begin{Highlighting}[]
\NormalTok{plot\_componentes }\OtherTok{\textless{}{-}} \ControlFlowTok{function}\NormalTok{(dados, parametros\_verdadeiros, parametros\_estimados, titulo) \{}
\NormalTok{  x }\OtherTok{\textless{}{-}} \FunctionTok{seq}\NormalTok{(}\DecValTok{0}\NormalTok{, }\FunctionTok{max}\NormalTok{(dados), }\AttributeTok{length.out =} \DecValTok{500}\NormalTok{)}
  
  \CommentTok{\# densidades teóricas e estimadas}
\NormalTok{  df }\OtherTok{\textless{}{-}} \FunctionTok{data.frame}\NormalTok{(}
    \AttributeTok{x =}\NormalTok{ x,}
    \AttributeTok{f1\_verdadeira =}\NormalTok{ parametros\_verdadeiros[}\StringTok{"delta"}\NormalTok{] }\SpecialCharTok{*} \FunctionTok{dgamma}\NormalTok{(x, parametros\_verdadeiros[}\StringTok{"alfa1"}\NormalTok{], parametros\_verdadeiros[}\StringTok{"beta1"}\NormalTok{]),}
    \AttributeTok{f2\_verdadeira =}\NormalTok{ (}\DecValTok{1} \SpecialCharTok{{-}}\NormalTok{ parametros\_verdadeiros[}\StringTok{"delta"}\NormalTok{]) }\SpecialCharTok{*} \FunctionTok{dgamma}\NormalTok{(x, parametros\_verdadeiros[}\StringTok{"alfa2"}\NormalTok{], parametros\_verdadeiros[}\StringTok{"beta2"}\NormalTok{]),}
    \AttributeTok{f1\_estimada =}\NormalTok{ parametros\_estimados[}\StringTok{"delta"}\NormalTok{] }\SpecialCharTok{*} \FunctionTok{dgamma}\NormalTok{(x, parametros\_estimados[}\StringTok{"alfa1"}\NormalTok{], parametros\_estimados[}\StringTok{"beta1"}\NormalTok{]),}
    \AttributeTok{f2\_estimada =}\NormalTok{ (}\DecValTok{1} \SpecialCharTok{{-}}\NormalTok{ parametros\_estimados[}\StringTok{"delta"}\NormalTok{]) }\SpecialCharTok{*} \FunctionTok{dgamma}\NormalTok{(x, parametros\_estimados[}\StringTok{"alfa2"}\NormalTok{], parametros\_estimados[}\StringTok{"beta2"}\NormalTok{])}
\NormalTok{  )}
  
  \FunctionTok{ggplot}\NormalTok{() }\SpecialCharTok{+}
    \FunctionTok{geom\_histogram}\NormalTok{(}\FunctionTok{aes}\NormalTok{(}\AttributeTok{x =}\NormalTok{ dados, }\AttributeTok{y =} \FunctionTok{after\_stat}\NormalTok{(density)),}
                   \AttributeTok{bins =} \DecValTok{40}\NormalTok{, }\AttributeTok{fill =} \StringTok{"grey85"}\NormalTok{, }\AttributeTok{color =} \StringTok{"white"}\NormalTok{, }\AttributeTok{alpha =} \FloatTok{0.6}\NormalTok{) }\SpecialCharTok{+}
    
    \FunctionTok{geom\_line}\NormalTok{(}\AttributeTok{data =}\NormalTok{ df, }\FunctionTok{aes}\NormalTok{(}\AttributeTok{x =}\NormalTok{ x, }\AttributeTok{y =}\NormalTok{ f1\_verdadeira, }\AttributeTok{color =} \StringTok{"C1 Verdadeira"}\NormalTok{),}
              \AttributeTok{linetype =} \StringTok{"dashed"}\NormalTok{, }\AttributeTok{linewidth =} \DecValTok{1}\NormalTok{) }\SpecialCharTok{+}
    \FunctionTok{geom\_line}\NormalTok{(}\AttributeTok{data =}\NormalTok{ df, }\FunctionTok{aes}\NormalTok{(}\AttributeTok{x =}\NormalTok{ x, }\AttributeTok{y =}\NormalTok{ f2\_verdadeira, }\AttributeTok{color =} \StringTok{"C2 Verdadeira"}\NormalTok{),}
              \AttributeTok{linetype =} \StringTok{"dashed"}\NormalTok{, }\AttributeTok{linewidth =} \DecValTok{1}\NormalTok{) }\SpecialCharTok{+}
    \FunctionTok{geom\_line}\NormalTok{(}\AttributeTok{data =}\NormalTok{ df, }\FunctionTok{aes}\NormalTok{(}\AttributeTok{x =}\NormalTok{ x, }\AttributeTok{y =}\NormalTok{ f1\_estimada, }\AttributeTok{color =} \StringTok{"C1 Estimada"}\NormalTok{),}
              \AttributeTok{linewidth =} \FloatTok{1.1}\NormalTok{) }\SpecialCharTok{+}
    \FunctionTok{geom\_line}\NormalTok{(}\AttributeTok{data =}\NormalTok{ df, }\FunctionTok{aes}\NormalTok{(}\AttributeTok{x =}\NormalTok{ x, }\AttributeTok{y =}\NormalTok{ f2\_estimada, }\AttributeTok{color =} \StringTok{"C2 Estimada"}\NormalTok{),}
              \AttributeTok{linewidth =} \FloatTok{1.1}\NormalTok{) }\SpecialCharTok{+}
    
    \FunctionTok{scale\_color\_manual}\NormalTok{(}
      \AttributeTok{name =} \StringTok{"Curvas de densidade"}\NormalTok{,}
      \AttributeTok{values =} \FunctionTok{c}\NormalTok{(}\StringTok{"C1 Verdadeira"} \OtherTok{=} \StringTok{"blue"}\NormalTok{,}
                 \StringTok{"C2 Verdadeira"} \OtherTok{=} \StringTok{"red"}\NormalTok{,}
                 \StringTok{"C1 Estimada"} \OtherTok{=} \StringTok{"darkblue"}\NormalTok{,}
                 \StringTok{"C2 Estimada"} \OtherTok{=} \StringTok{"darkred"}\NormalTok{)}
\NormalTok{    ) }\SpecialCharTok{+}
    
    \FunctionTok{theme\_minimal}\NormalTok{() }\SpecialCharTok{+}
    \FunctionTok{labs}\NormalTok{(}\AttributeTok{title =}\NormalTok{ titulo, }\AttributeTok{x =} \StringTok{"x"}\NormalTok{, }\AttributeTok{y =} \StringTok{"Densidade"}\NormalTok{) }\SpecialCharTok{+}
    \FunctionTok{theme}\NormalTok{(}\AttributeTok{plot.title =} \FunctionTok{element\_text}\NormalTok{(}\AttributeTok{size =} \DecValTok{12}\NormalTok{, }\AttributeTok{face =} \StringTok{"bold"}\NormalTok{),}
          \AttributeTok{legend.title =} \FunctionTok{element\_text}\NormalTok{(}\AttributeTok{size =} \DecValTok{10}\NormalTok{),}
          \AttributeTok{legend.text =} \FunctionTok{element\_text}\NormalTok{(}\AttributeTok{size =} \DecValTok{9}\NormalTok{))}
\NormalTok{\}}


\NormalTok{pairs }\OtherTok{=} \FunctionTok{c}\NormalTok{(}\AttributeTok{mfrow=}\FunctionTok{c}\NormalTok{(}\DecValTok{1}\NormalTok{,}\DecValTok{3}\NormalTok{))}
\FunctionTok{plot\_componentes}\NormalTok{(dados1, parametros1, resultado1}\SpecialCharTok{$}\NormalTok{estimativas, }\StringTok{"Cenário 1"}\NormalTok{)}
\end{Highlighting}
\end{Shaded}

\pandocbounded{\includegraphics[keepaspectratio]{lista3_MCCDII_files/figure-latex/ex7-1.pdf}}

\begin{Shaded}
\begin{Highlighting}[]
\FunctionTok{plot\_componentes}\NormalTok{(dados2, parametros2, resultado2}\SpecialCharTok{$}\NormalTok{estimativas, }\StringTok{"Cenário 2"}\NormalTok{)}
\end{Highlighting}
\end{Shaded}

\pandocbounded{\includegraphics[keepaspectratio]{lista3_MCCDII_files/figure-latex/ex7-2.pdf}}

\begin{Shaded}
\begin{Highlighting}[]
\FunctionTok{plot\_componentes}\NormalTok{(dados3, parametros3, resultado3}\SpecialCharTok{$}\NormalTok{estimativas, }\StringTok{"Cenário 3"}\NormalTok{)}
\end{Highlighting}
\end{Shaded}

\pandocbounded{\includegraphics[keepaspectratio]{lista3_MCCDII_files/figure-latex/ex7-3.pdf}}

\end{document}
